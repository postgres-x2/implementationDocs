%
% Chapter "Configure Database"
%
% Description of initialization and configuring database.
%


%========= SECTION SECTION ===================================================

\section{\label{sec:initdb}Changes in initdb}

  Essentially, we have two extension/modification to \file{initdb} utility:
  
  \begin{itemize}
	  \item Setting up its own node name to \file{pgxc_node} catalog.
	  \item Vacuum freeze every template and intial database: \file{template0}, \file{template1} and
			\file{postgres}.
  \end{itemize}
  
  Other extensions in the catalog, built-in functions and other embedded database objects specific to
  \XC{} are in catalog initialization and initdb does not need specific change for this.
  
  The background of the need of complete vacuum freeze comes from dynamic node addition.
  
  In original initdb, it comsumes some XIDs locally.
  Although initdb cannot run with GTM so far, it was not an issue if all the nodes are
  initialized at the same time.
  When GTM works for the first time, it can begin with a GXID larger than the last XID
  value consumed by initdb.
  \file{-x} option worked to deal with this situation.
  
  When a new node should be added dynamically, current GXID could be of any value.
  It can be just before GXID wraps around!
  To enable new node (initialized with \file{initdb}) to be added to running cluster,
  it was essential to vacuum freeze the database completeley before joining \XC{} clusetr
  so that it can begin with any GXID value.
  
  Following shows inddividual extension to \file{initdb}.
  Please note that \file{initdb} source file is essentially one.
  \file{findtimezone.c} and \file{po} directory are common to \XC{} which does not need any
  modification.


%------- Subsec Subsec -----------------------------------------------------

\subsection{New option}

  \file{--nodename} option specifies its node name.
  \file{initdb} itself does not require if the new node should be a coordinator or a datanode.
  It is specified with \file{-Z} option of \file{pg_ctl} utility, although it is not a good idea
  to switch between coordinator and datanode while the node is already in use.
  
  You will find this change in \file{main()} and \file{setup_config()} functions.


%------- Subsec Subsec -----------------------------------------------------

\subsection{Vacuum freeze}

  Vacuum freeze feature is implemented as \file{vacuumfreeze()} function.
  This is done by running postgres binary against new initialized database cluster and issue
  \texttt{VACUUM FREEZE} command.

  Please note that postgres binary need to run with local XID in this case because GTM might
  not have configured and running yet.
  For this purpose, \file{initdb} runs \file{postgres} binary against the new database
  with ``\file{--single}'' and ``\file{--localxid}'' options.
  Former option specifies \file{postgres} to run as a single user mode and the latter
  specifies to run with local XID, which is \XC-specific extension to \file{postgres}
  binary.

  \file{vacuumfreeze()} function is called from \file{main()} function.


%========= SECTION SECTION ===================================================

\section{\label{sec:pgconf}Extension of postgresql.conf}


%------- Subsec Subsec -----------------------------------------------------

\subsection{List of additional GUC parameters}

  All the GUC parameters specific to \XC{} will be found at \file{guc.c} module.
  
  \GUC{enable_remotejoin}{boolean}{true}
  
    Enables join push-down to remote node.
    This should be turned on all the time unless you are debugging \XC{} planner or executor.
  
  \GUC{enable_remotejoin}{boolean}{true}
  
    Enables fast query shipping.
    This should be turned on all the time unless you are debugging \XC{} planner or executor.
  
  \GUC{enable_remotegroup}{boolean}{true}
  
    Enables \texttt{GROUP BY} pushdown.
    This shoudl be turned on all the time unless you are debugging \XC{} planner or executor.
  
  \GUC{enable_remotesort}{boolean}{true}
  
    Enables \texttt{ORDER BY} pushdonw.
    This should be turned on all the time unless you are debugging \XC{} planner or executor.
  
  \GUC{enable_remotelimit}{boolean}{true}
  
    Enables \texttt{LIMIT} clause pushdown.
    This should be turned on all the time unless you are debugging \XC{} planner or executor.
  
  \GUC{gtm_backup_barrier}{boolean}{false}
  
    Controls if GTM backs up restart point for each \texttt{BARRIER}.
  
  \GUC{persistent_datanode_connections}{boolean}{false}
  
    If on, the pooler will not release the acquired connec tion.
  
  \GUC{enforce_two_phase_commit}{boolean}{true}
  
    This is mainly for the compatibility and consistency in the regression test.
    If set to \file{false}, implicit two phase commit will not be used.
  
  \GUC{xc_maintenance_mode}{boolean}{false}
  
    When \file{true}, it enables \texttt{EXECUTE DIRECT} to perform write operation.
    This cannot be turned on in \texttt{postgresql.conf}.
    Only a superuser can tur this on using \texttt{SET} command.
    It is intended to be used to maintain cluster-wide data consistency.
    It is used in \file{pgxc_clean}.
  
  \GUC{require_replicated_table_pkey}{boolean}{true}
  
    This congtrols replicated table update.
    If no primary key or other unique constraint is not available in replicated table update and
	needs ctid for update, this option make such statment to fail, when turned on.
    To maintain replicated table consisiten, you should not turn this off unless you fully
	understand its outcome and you intend to do so.
  
  \GUC{min_pool_size}{integer}{1}
  
    Specifies minimum number of pooled connection to datanodes.
  
  \GUC{max_pool_size}{integer}{100}
  
    Specifies maximum number of pooled connection to datanodes.
  
  \GUC{pooler_port}{integer}{6667}
  
    Port number for pooler to listen.
  
  \GUC{gtm_port}{integer}{6666}
  
    Port number of GTM or GTM proxy to connect to.
    If the node (coordinator or datanode) is connecting to GTM directly, specify GTM port number.
    If conneted through GTM proxy, specify GTM proxy port number.
  
  \GUC{max_datanodes}{integer}{16}
  
    Maximum number of datanodes.
  
  \GUC{max_coordinators}{integer}{16}
  
    Maximum number of coordinators.
  
  \GUC{gtm_host}{string}{none}
  
    Host name of GTM or GTM proxy connecting.
  
  \GUC{pgxc_node_name}{string}{none}
  
    GTM node name.
  
  \GUC{remotetype}{enum}{``application''}
  
    Not used at present.


%------- Subsec Subsec -----------------------------------------------------

\subsection{Additional functions to handle GUC parameters}

  \file{check_pgxc_maintenance_mode()} function was added to disable \file{xc_maintenance_mode}
  GUC turned on at \file{postgresql.conf} file.
  This function checks if the parameter value was set to \file{true} in \file{postgresql.conf}.
