%%%%%%%%%%%%%%%%%%%%%%%%%%%%%%%%%%%%%%%%%%%%%%%%%%%%%%%%%%%%%%%%%%%%%%%%%%%%%%%%
%
% Backend Transaction Management
%
%%%%%%%%%%%%%%%%%%%%%%%%%%%%%%%%%%%%%%%%%%%%%%%%%%%%%%%%%%%%%%%%%%%%%%%%%%%%%%%%

%========= SECTION SECTION ===================================================

\section{Transaction Management at coordinator/datanode backends}

  Each coordinator/datanode backend needs to connect to GTM to obtain Global Transaction
  Id (\texttt{GXID}) and global snapshot.
  \file{gtm.c} module in \file{src/backend/access/transam} takes care of connection and
  communication between each backend and GTM.
  
  This section describes functions defined in \file{gtm.c} and other dedicated transaction
  handling related to GTM in coordinator/datanode backend processes.
  

%------- Subsec Subsec -----------------------------------------------------

\subsection{\texttt{gtm.c} module}
  
  \file{gtm.c} module handles connection and communication from coordinator/datanode
  backend and gtm/gtm\_proxy.
  
  Functions defined in this module are as follows:
  
  \FUNC{IsGTMConnected()}
  
      This function checks if connection to GTM is alive.
	  This is called from the following code:
      
	  % Function cross reference
      \FuncRefHdr
		  \FuncRef{AtEOXact_GlobalTxn()}{xact.c}{
			Used to determine what transaction ID should be used at the end of the
			transaction.
		  }\\ \hline
      \FuncRefTrailor
  
  
  \FUNC{CheckConnection()}
  
      This function checks if a connection to GTM has been established.
	  If not, it establishes a connection to GTM.
      
      This is a static function and is used only in \file{gtm.c} locally.
  
  \FUNC{InitGTM()}
  
      This function establishes a connection to GTM 
      and is used only within \file{gtm.c} at present.
  
  \FUNC{CloseGTM()}
  
      This function closes a connection to GTM
	  This is called from the following code:
      
	  % Function cross reference
      \FuncRefHdr
		  \FuncRef{PGXCNodeCleanAndRelease()}{execRemote.c}{
			Called when the backend is ending.
		  }\\ \hline
      \FuncRefTrailor
      
      This function is also called within \file{gtm.c} module internally.
  
  \FUNC{BeginTranGTM()}
  
      This function informs GTM of the start of a new transaction and obtains
      global transaction ID (\texttt{GXID}).
      
      This function is called from the following codes:
      
	  % Function cross reference
      \FuncRefHdr
		  \FuncRef{GetNewTransactionId()}{varsup.c}{
			Called to obtain global transaction ID.
		  }\\
		  \FuncRef{GetAuxilliaryTransactionId()}{xact.c}{
			Used to set auxilliaryTransactionId entry to CurrentTransactionState.
		  }\\ \hline
      \FuncRefTrailor
  
  \FUNC{BeginTranAutovacuumGTM()}
  
      This function is similar to \file{BeginTranGTM()} but only for autovacuum process.
      GXID for autovacuum process does not appear in the global snapshot.
      
      This function is called from the following codes:
      
      \FuncRefHdr
		  \FuncRef{GetNewTransactionId()}{varsup.c}{
			Called when obtaining XID for autovacuum process.
		  }\\ \hline
      \FuncRefTrailor
  
  \FUNC{CommitTranGTM()}
  
      This function tells GTM the specified transaction is committed and sets
      current transaction ID to invalid value.
      
      This is called from the following codes:
      
	  % Function cross reference
      \FuncRefHdr
		  \FuncRef{AtEOXact\_GlobalTxn()}{xact.c}{
			Called at the end of global transaction.
			It is called only when it is needed to close the transaction on the GTM
		  }\\ \hline
      \FuncRefTrailor
  
  \FUNC{CommitPreparedTranGTM()}
  
      This function tells GTM to commit a prepared transaction.
      
      This is called from the following codes:
      
	  % Function cross reference
      \FuncRefHdr
		  \FuncRef{AtEOXact_GlobalTxn()}{xact.c}{
			Used to mark the end of global transaction.
		  }\\
		  \FuncRef{FinishRemotePreparedTransaction()}{execRemote.c}{
			Used to finish prepared transaction at remote nodes.
		  }\\ \hline
      \FuncRefTrailor
  
  \FUNC{RollbackTranGTM()}
  
      This function tells GTM the specified transaction is aborted and
      sets current transaction ID to invalid value.
      
      This is called from the following codes:
      
	  % Function cross reference
      \FuncRefHdr
		  \FuncRef{AtEOXact_GlobalTxn()}{xact.c}{
			Called to mark the end of global transaction.
		  }\\
		  \FuncRef{FinishRemotePreparedTransaction()}{execRemote.c}{
			Called to finish prepared transaction at remote nodes.
		  }\\ \hline
      \FuncRefTrailor
  
  \FUNC{StartPreparedTranGTM()}
  
      This function tells GTM that prepare transaction commands starts with remote nodes.
      
      This is called from the following codes:
      
      \FuncRefHdr
		  \FuncRef{PreAbort_Remote()}{execRemote.c}{
			Called to abort remote truncations.
		  }\\
		  \FuncRef{PostPrepare_Remote()}{execRemote.c}{
			Called in post-prepare handling in remote nodes.
		  }\\ \hline
      \FuncRefTrailor
  
  \FUNC{PrepareTranGTM()}
  
      This function tells GTM that prepare transaction commands was successful.
      
      This is called from the following codes:
      
	  % Function cross reference
      \FuncRefHdr
		  \FuncRef{PreAbort_Remote()}{execRemote.c}{
			Called at re-abort handling.
		  }\\
		  \FuncRef{PostPrepare_Remote()}{execRemte.c}{
			Called at post-prepare handling.
		  }\\ \hline
      \FuncRefTrailor
  
  \FUNC{GetGIDDataGTM()}
  
      This function obtains GTM internal information of fresh GXID, GXID of the prepared
	  transaction, and datanode/coordinator node list involved in the prepare.
      
      This is for the future use.
  
  \FUNC{GetSnapshotGTM()}
  
      This function obtains global snapshot from GTM.
      
      This is called from the following codes:
      
	  % Function cross reference
      \FuncRefHdr
		  \FuncRef{GetSnapshotDataDataNode()}{procarray.c}{
			Called to get snapshot data for datanode.
		  }\\
		  \FuncRef{GetSnapshotDataCoordinator()}{procarray.c}{
			Called to get snapshot data for coordinator.
		  }\\ \hline
      \FuncRefTrailor
  
  \FUNC{CreateSequenceGTM()}
  
      This function creates a sequence on GTM.
      
      This is called from the following codes:
      
	  % Function cross reference
      \FuncRefHdr
		  \FuncRef{DefineSequence()}{sequence.c}{
			Called in creating sequence.
		  }\\ \hline
      \FuncRefTrailor
  
  \FUNC{AlterSequenceGTM()}
  
      This function alters a sequence on GTM.
      
      This is called from the following code:
      
	  % Function cross reference
      \FuncRefHdr
		  \FuncRef{AlterSequence()}{sequence.c}{
			Called to modify the definition of a sequence.
		  }\\ \hline
      \FuncRefTrailor
  
  \FUNC{GetNextValGTM()}
  
      This function gets the next sequence value.
      
      This is called from the following codes:
      
	  % Function cross reference
      \FuncRefHdr
      \FuncRef{nextval\_internal()}{sequence.c}{
      	Called to get the next value of the sequence.
      }\\ \hline
      \FuncRefTrailor
  
  \FUNC{SetValGTM()}
  
      This functions sets the value of the sequence.
      
      This is called form the following codes:
      
	  % Function cross reference
      \FuncRefHdr
		  \FuncRef{do_setval()}{sequence.c}{
		  This is called in handling 2 and 3 argument forms of {\tt SETVAL}.
		  }\\ \hline
      \FuncRefTrailor
  
  \FUNC{DropSequenceGTM()}
  
      This function drops sequence depending on the key type.
      
      This is called from the following codes:
      
	  % Function cross reference
      \FuncRefHdr
		  \FuncRef{drop_sequence_cb()}{sequence.c}{
			Called in a callback of sequence drop.
		  }\\
		  \FuncRef{dropdb()}{dbcommands.c}{
			Called in dropping a database.
		  }\\ \hline
      \FuncRefTrailor
  
  \FUNC{RenameSequenceGTM()}
  
      This function renames sequence on GTM.
      
      This is called from the following codes:
      
	  % Function cross reference
      \FuncRefHdr
		  \FuncRef{RenameRelationInternal()}{tablecmds.c}{
			Called in changing the name of a relation.
		  }\\
		  \FuncRef{AlterTableNamespaceInternal()}{tablecmds.c}{
			This is called in relocating a table or materialized view to another namespace.
		  }\\
		  \FuncRef{AlterSeqNamespaces()}{tablecmds.c}{
			This is called to move all {\tt SERIAL} column sequences of the specified relation
			to another namespace.
		  }\\
		  \FuncRef{rename_sequence_cb()}{sequence.c}{
			Called in sequence rename callback.
		  }\\
		  \FuncRef{doRename()}{dependency.c}{
			Called in renaming the given object.
		  }\\ \hline
      \FuncRefTrailor
  
  \FUNC{RegisterGTM()}
  
      This function registers the specified node to GTM.
      Connection for registering is used just once the closed.
      
      This is called from the following codes:
      
	  % Function cross reference
      \FuncRefHdr
		  \FuncRef{sigusr1_handler()}{postmaster.c}{
			Called in handling signal conditions from child processes.
		  }\\ \hline
      \FuncRefTrailor
  
  \FUNC{UnregisterGTM()}
  
      This function unregisters the given node from GTM.
      Connection for registering is used just once the closed.
      
      This is called form the following codes:
      
	  % Function cross reference
      \FuncRefHdr
		  \FuncRef{pmdie()}{postmaster.c}{
			Called in a signal handler to handle various postmaster signals.
		  }\\
		  \FuncRef{sigusr1_handler()}{postmaster.c}{
			Called in handling signal conditions from child processes.
		  }\\ \hline
      \FuncRefTrailor
  
  \FUNC{ReportBarrierGTM()}
  
      This function reports barrier to GTM.
      This is used to backup GTM restart point for given barrier id.
      
	  % Function cross reference
      \FuncRefHdr
		  \FuncRef{RequestBarrier()}{barrier.c}{
			Called in handling \texttt{CREATE BARRIER} statement.
		  }\\ \hline
      \FuncRefTrailor


%------- Subsec Subsec -----------------------------------------------------

\subsection{\texttt{xact.c} module}

  This module has \XC-specific functions as follows:
  
  \FUNC{RegisterTransactionLocalNode()}
  
      Marks if the local node has done some write activity.
      
      This is called form the following codes:
      
	  % Function cross reference
      \FuncRefHdr
		  \FuncRef{ExecRemoteUtility()}{execRemote.c}{}\vspace{-10pt} \\ \hline
      \FuncRefTrailor
  
  \FUNC{ForgetTransactionLocalNode()}
  
      Forgets about the local node's involvement in the transaction.
      
      Called from the following codes:
      
	  % Function cross reference
      \FuncRefHdr
		  \FuncRef{CommitTransaction()}{xact.c}{}\vspace{-10pt} \\
		  \FuncRef{PrepareTransaction()}{xact.c}{}\vspace{-10pt} \\
		  \FuncRef{AbortTransaction()}{xact.c}{}\vspace{-10pt} \\ \hline
      \FuncRefTrailor
  
  \FUNC{IsTransactionLocalNode()}
  
      Checks if the local node is involved in the transaction.
      
      Called form the following codes:
      
      It is for the future use and is not used at present.
  
  \FUNC{IsXidImplicit()}
  
      Checks if the given xid is for implicit 2PC.
      
      Called form the following code:
      
	  % Function cross reference
      \FuncRefHdr
		  \FuncRef{standard_ProcessUtility()}{utility.c}{
			Called in handling \texttt{PREPARE TRANSACTION} command.
		  }\\ \hline
      \FuncRefTrailor
