%%%%%%%%%%%%%%%%%%%%%%%%%%%%%%%%%%%%%%%%%%%%%%%%%%%%%%%%%%%%%%%%%%%%%%%%%%%%%%%%
%
% pgxc_monitor module description
%
%%%%%%%%%%%%%%%%%%%%%%%%%%%%%%%%%%%%%%%%%%%%%%%%%%%%%%%%%%%%%%%%%%%%%%%%%%%%%%%%

  {\tt pgxc\_monitor} is \XC{} utility to check if specified node is running.
  
  Reference document will be found at
  \url{http://postgres-xc.sourceforge.net/docs/1_2_1/pgxcmonitor.html}.
  
  Its source code is located at \file{contrib/pgxc_monitor}.
  
  General description how to monitor each component is given below.
  Please note that \file{pgxc_monitor} does not include retry if it couldn't determine
  the target is running.
  Applications should do it.


%------- Subsec Subsec -----------------------------------------------------

\subsection{Monitoring GTM and GTM Proxy}

  It is performed by connecting to specified GTM and GTM proxy using GTM client library
  function \file{PQconnectGTM()}.
  It determines that specified GTM or GTM Proxy is running if connection is established.


%------- Subsec Subsec -----------------------------------------------------

\subsection{Monitoring Coordinator and Datanode}

  It is performed by issuing ``\texttt{SELECT 1;}'' SQL statement through \file{psql}.


