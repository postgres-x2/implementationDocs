
%------- Subsec Subsec -----------------------------------------------------

\subsection{Trigger Syntax}

  Trigger syntax is defined in \file{gram.y}.
  There is no \XC-specific change in trngger syntax.


%------- Subsec Subsec -----------------------------------------------------

\subsection{Creating Trigger}

  \texttt{CREATE TRIGGER} statement is parsed into \file{CreateTrigStmt} structure and passed to
  \file{ProcessUtilitySlow()} function in \file{utility.c}.
  Here, after local handling has been done, the statement is propagated to other nodes where the base
  relation is defined.
  
  \file{commands/trigger.c} implements most of trigger DDL handler.
  \file{CreateTrigger()} is the main handler of {\tt C\tt REATE TRIGGER} statement.
  They have no \XC-specific changes.


%------- Subsec Subsec -----------------------------------------------------

\subsection{Changing Trigger definition}

  It is handled by \file{ExecRenameStmt()} in \file{alter.c}.
  Before this, the statement is propagated to other nodes where the base relation is defined
  \footnote{
	  We need to check if \texttt{ALTER TABLE} ... \texttt{ADD NODE} handle this correctly.
  }.
  
  In the case of \texttt{ALTER TRIGGER}, it is then passed to \file{remanetrig()} in \file{trigger.c}.
  
  They have no \XC-specific change.


%------- Subsec Subsec -----------------------------------------------------

\subsection{Dropping Trigger}

  This statement is handled by \file{ExecDropStmt()} in \file{utility.c} and then passed to
  \file{RemoveObjects()} in \file{dropcmds.c} before it is propagated to other nodes.
  
  \file{RemoveObjects()} does not have \XC-specific changes.


%------- Subsec Subsec -----------------------------------------------------

\subsection{Firing Trigger}

  Most of the changes needed to support triggers are in firing triggers, implemented in
  \file{trigger.c}.
  
  This section describes \XC-specific utility functions in this module and then describes
  changes to existing trigger firing functions.


% - - - - Func - - - - - - -  - - - - - - - - - - - - - - - - - - - - -

\FUNC{pgxc_should_exec_triggers()}

  Determines if all of the triggers for the relation should be executed
  here, on this node.
  On a coordinator, returns true if there is at least one non-shippable trigger for the relation
  that matches the given event, level and timing.
  (or for any local-only table for that matter), returns false if all of the matching triggers
  are shippable.
  
  PG behaviour is such that the triggers for the same table should be executed in
  alphabetical order.
  This make it essential to execute all the triggers on the same node, be it coordinator or
  datanode.
  So the idea used here is: if all matching triggers are shippable, they should be executed on local
  tables (i.e. on datanodes).
  Even if there is at least one single trigger that is not shippable, all the triggers should be fired on remote
  tables (i.e. on the coordinator).
  This ensures that either all the triggers are executed on coordinator, or all are executed on
  datanodes.


% - - - - Func - - - - - - -  - - - - - - - - - - - - - - - - - - - - -

\FUNC{pgxc_is_trigger_firable()}

This function is defined only to handle the special case if the trigger is an internal trigger.
Once we support global constraints, we should not handle this as a special case:
global constraint triggers would be executed just like normal triggers.
Internal triggers are internally created triggers for constraints such as foreign key or
unique constraints.
Currently we always execute an internal trigger on datanodes, assuming that the constraint
trigger function is always shippable to datanodes.
We can safely assume so because we disallow constraint creation for scenarios where the
constraint needs access to records on other nodes.


% - - - - Func - - - - - - -  - - - - - - - - - - - - - - - - - - - - -

\FUNC{pgxc_is_internal_trig_firable()}

  This function determines if a given internal trigger is firable at this node.


% - - - - Func - - - - - - -  - - - - - - - - - - - - - - - - - - - - -

\FUNC{pgxc_get_trigger_tuple()}

  Obtains tuple of the trigger target.


% - - - - Func - - - - - - -  - - - - - - - - - - - - - - - - - - - - -

\FUNC{pgxc_check_distcol_update()}

  Compares the distribution column values given to the function and error out if they
  are different.
  This is called to make sure triggers have not updated the distribution column.


% - - - - subsubsection - - - - - - -  - - - - - - - - - - - - - - - - - - - - -

\subsubsection{Other Trigger-firing Functions}

Other trigger-firint functions are modofied to determine if the trigger should be fired
in this node.
Fire it if yes.

