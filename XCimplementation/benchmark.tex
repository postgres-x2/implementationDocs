%
% This file includes description of benchmark changes in Postgres-XC.
%

%========= SECTION SECTION ===================================================

\section{\label{sec:dbtOne}DBT-1 Benchmark}

  \XC{} uses DBT-1 as basic benchmark test to measure its performance for each build.
  As described in the section~\ref{arch:dbt1} at page~\pageref{arch:dbt1}, 
  distribution or replication option was added to each table based on
  its characteristics of the size, update
  frequency, and join operation with others.
  
  Outline of the table design is illustrated in Figure~\ref{archfig:15} in
  page~\pageref{archfig:15}.
  
  Modified DBT-1 source code will be found in sourceforge reporitory with the url as
  \url{git://git.code.sf.net/p/postgres-xc/dbt1 postgres-xc-dbt1}.
  
  Other major modification to the benchmark is as follows:
  
  \begin{itemize}		% Major modification in DBT-1 test.
  	\item \file{author} table is replicated (\file{create_table.sql}).
  	\item \file{country} table is replicated (\file{create_table.sql}).
  	\item \file{address} table has additional column \file{addr_c_id} for its customer ID and
			is hash-distributed using \file{addr_c_id} (\file{create_table.sql}).
  	\item \file{customer} table is hash-distributed using \file{c_id} (customer ID)
			(\file{create_tables.sql}).
  	\item \file{item} table is replicated and does not have \file{i_stock} column
			(\file{create_tables.sql}).
  	\item \file{i_stock} column of \file{item} table is now in the new table \file{stock},
			which is hash-distributed using \file{st_i_id} (stock item id) (\file{create_tables.sql}).
  	\item \file{st_i_id} column is the primary key of \file{stock} table (\file{create_indexes.sql}).
  	\item \file{orders} table is hash-distributed using \file{o_c_id} column (order cumstomer id)
			(\file{create_tables.sql}).
  	\item (\file{o_id}, \file{o_c_id}) is the primary key for \file{orders} table
			(\file{create_indexes.sql}).
  	\item \file{orders} table has additional index over \file{i_o_c_id} column.
  	\item \file{order_line} table has additional column \file{ol_c_id} (order line cumstomer id) and
			is hash-distributed using \file{ol_c_id} column (\file{create_tables.sql}).
  	\item (\file{ol_o_id}, \file{ol_id}, \file{ol_c_id}) is the primary key for \file{order_line}
			table (\file{create_indexes.sql}).
  	\item \file{cc_xacts} table has additional column \file{cx_c_id} (cc customer id), which is
			hash-distributed using \file{cx_c_id} (\file{create_tables.sql}).
  	\item (\file{cx_o_id}, \file{cx_c_id}) is the primary key of \file{cc_xact} table
			(\file{create_indexes.sql}).
  	\item \file{shopping_cart} table is hash-distributed using \file{sc_id} (shopping cart id).
  	\item \file{shopping_cart_line} table has additional column \file{scl_c_id}
		(shopping cart customer id), which is hash-distributed using \file{scl_sc_id}
		(shopping cart id) (\file{create_tables.sql}).
  	\item Foreign key from \file{adress} table to \file{customer} table was modified to just
			an index (\file{create_fk.sql}).
  	\item Foreign key from \file{order_line} table to \file{orders} table was modified
  		(\file{create_fk.sql}).
  	\item Foreign key from \file{stock} table to \file{item} table was added.
  	\item Message modified to reflect changes in table deesign.
  		Modifications are indicated by comments with ``\file{pgxc}.''
  	\item SQL statement modified to reflect changes in table design.
  		Modifications are indicated by comments with ``\file{pgxc}.''
  	\item Original ODBC interface to the database was changed to libpq
		(\file{libpq_com} and \file{inc_psql_libpq} directories).
  	\item \file{dbdriver} files are modified according to the interface change from ODBC to libpq.
   		Changes are indicated with \file{\#ifdef} directive using \file{LIBPQ} (\file{eu.c} and \file{main.c}).
  \end{itemize}

%========= SECTION SECTION ===================================================

\section{\label{sec:pgbench}Pgbench}

  Modification to pgbench benchmark is as follows:
  
  \begin{itemize}
	  \item For write benchmark, each table was distributed by modulo using bid.
	  		The backend to choose modulo, not hash, is that the number of bid value is relatively
			small and it is not easy to have a good balance of each datanode workload.
	  \item For read benchmark, each table was replicated.
  \end{itemize}

  The above changes are reasonable because it is a commo practice to have different
  table design and arrangement for different workloads.

