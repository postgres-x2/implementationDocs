%%%%%%%%%%%%%%%%%%%%%%%%%%%%%%%%%%%%%%%%%%%%%%%%%%%%%%%%%%%%%%%%%%%%%%%%%%%%%%%%
% Chapter "Cluster Management"
%
% Description of cluster management SQL commands and related tables.
%
%%%%%%%%%%%%%%%%%%%%%%%%%%%%%%%%%%%%%%%%%%%%%%%%%%%%%%%%%%%%%%%%%%%%%%%%%%%%%%%%


%========= SECTION SECTION ===================================================

\section{\label{sec:clusterNodeInfo}Cluster Node}


%------- Subsec Subsec -----------------------------------------------------

\subsection{Cluster management statement}

  To manage the cluster node, there are \texttt{CREATE NODE}, \texttt{ALTER NODE}, and
  \texttt{DROP NODE} statements.
  
  \XC{} added the symtax to implement these cluster management statement.
  The grammar is defined in \file{src/backend/parser/gram.y}.
  Additional changes were made to
  support these statements.
  The changes are listed in Table \ref{tab:chgnodemgmtstmt}.
  
  {
	  \footnotesize
	  \begin{table}[htp]
		  \begin{center}
			  \caption{\label{tab:chgnodemgmtstmt}Changes to support node management statements}
			  \begin{tabular}{llp{0.4\hsize}} \hline
				  File & Function name & Description \\ \hline
				  \file{utility.c} & \file{standard_ProcessUtility()} & {\raggedright
							Changed to recognize \file{CreateNodeStmt}, \file{AlterNodeStmt} and \file{DropNodeStmt}
							type statements as utility statements and call appropriate functions. }\\ 
				  \file{copyfunc.c} & \file{CopyObject()} & {\raggedright
							Added support for \file{CreateNodeStmt}, \file{AlterNodeStmt} and \file{DropNodeStmt}.}\\
				  \file{equalfunc.c} & \file{equal()} & {\raggedright
							Added support for \file{CreateNodeStmt}, \file{AlterNodeStmt} and \file{DropNodeStmt}.}\\
				  \file{utility.c} & \file{IsStmtAllowedInLockMode()} & {\raggedright
							Allowed to execute \file{CreateNodeStmt}, \file{AlterNodeStmt} and \file{DropNodeStmt} in lock mode.}\\
				  \file{utility.c} & \file{CreateCommandTag()} & {\raggedright
							Added support for \file{CreateNodeStmt}, \file{AlterNodeStmt} and \file{DropNodeStmt}.}\\
				  \hline
			  \end{tabular}
		  \end{center}
	  \end{table}
  }
  
  Changes are simple: add new case branches to \file{switch}-\file{case} block for node tags and write a code for them.
  
  % Program Listing ----------------------------------------
  \begin{lstlisting}
case T_CheckPointStmt:
	retval = (void *) makeNode(CheckPointStmt);
	break;
#ifdef PGXC
case T_BarrierStmt:
	retval = _copyBarrierStmt(from);
	break;
case T_AlterNodeStmt:
	retval = _copyAlterNodeStmt(from);
	break;
case T_CreateNodeStmt:
  \end{lstlisting}
  % End program listing ----------------------------------
  
  The implementation of cluster management statement functions is described later.


% - - - - subsubsection - - - - - - -  - - - - - - - - - - - - - - - - - - - - -

\subsection{Node information catalog}

  Node management statements manipulates the system catalog \file{pgxc_node}.
  The definition of the system catalog is given below.
  You can find the column in the catalog corresponding to the options of node management statements.
  
  % Prigram Listing ----------------------------------------
  \begin{lstlisting}
                         Table "pg_catalog.pgxc_node"
      Column      |  Type   | Modifiers | Storage | Stats target | Description 
------------------+---------+-----------+---------+--------------+-------------
 node_name        | name    | not null  | plain   |              | 
 node_type        | "char"  | not null  | plain   |              | 
 node_port        | integer | not null  | plain   |              | 
 node_host        | name    | not null  | plain   |              | 
 nodeis_primary   | boolean | not null  | plain   |              | 
 nodeis_preferred | boolean | not null  | plain   |              | 
 node_id          | integer | not null  | plain   |              | 
Indexes:
    "pgxc_node_id_index" UNIQUE, btree (node_id), tablespace "pg_global"
    "pgxc_node_name_index" UNIQUE, btree (node_name), tablespace "pg_global"
    "pgxc_node_oid_index" UNIQUE, btree (oid), tablespace "pg_global"
Has OIDs: yes
Tablespace: "pg_global"
  \end{lstlisting}
  % End program listing ----------------------------------
  
  This system catalog is defined in \file{src/include/catalog/pgxc_node.h},
  which is created by \file{initdb} through the BKI file.
  To get the node's attribute easily, utility functions listed in Table
  \ref{tab:nodeattrfunc} are implemented in \file{src/backend/utils/cache/lsyscache.c}.
  For the more high level use, please see the following \textbf{Node Manager} subsection.
  
  \begin{table}[htp]
	  \begin{center}
		  \caption{\label{tab:nodeattrfunc}Node attribute query functions}
		  \begin{tabular}{lp{0.5\hsize}} \hline
			  Function name & Description \\ \hline
			  \file{get_pgxc_nodeoid()} & Get the node OID by node name. \\
			  \file{get_pgxc_nodename()} & Get the node name by OID. \\
			  \file{get_pgxc_node_id()} & Get the node ID by OID. \\
			  \file{get_pgxc_nodetype()} & Get the node type by OID. \\
			  \file{get_pgxc_nodehost()} & Get the node host by OID. \\
			  \file{get_pgxc_nodeport()} & Get the node port by OID. \\
			  \file{is_pgxc_nodepreferred()} & Get whether the node is preferred by OID. \\
			  \file{is_pgxc_nodeprimary()} & Get whether the node is primary by OID. \\
			  \hline
		  \end{tabular}
	  \end{center}
  \end{table}



%------- Subsec Subsec -----------------------------------------------------

\subsection{Node manager}

  The node manager is implemented in \file{nodemgr.c}.
  This subsection describes the APIs of the node manager.
  
  The node manager consists of two types functions as follows:

  \begin{itemize}
	  \item Gets node information from the shared memory where the information in
			\file{pgxc_node} catalog are copied.
	  \item Manipulates node information in the \file{pgxc_node} catalog.
  \end{itemize}
  
  Please note that information in the shared memory and the system catalog are not
  always synchronized.
  To update information in the shared memory, a program need to call specific API.
  The pooler uses the shared memory only and both information are used by the
  database/tablespace size calculator and the advisory lock.
  
  \FUNC{NodeTablesShmemInit()}	%--------------------------------------
  
    Creates or attaches the shared memory for the node table.
    
    This function creates or attaches the shared memory for the node table.
    The shared memory for the node table has fixed size.
    It means that the maximum size of the memory is allocated when it is initialized.
    
    This function is called in the initialization sequence of the process.
    It is needed both for \file{postmaster} and other various processes like backends etc.
    
    This is called from the following codes:
    
    \FuncRefHdr
		\FuncRef{CreateSharedMemoryAndSemaphores}{ipci.c}{
		  Used in creating and attaching various shared memories.
		}\\ \hline
    \FuncRefTrailor
  
  \FUNC{NodeTablesShmemSize()}	%--------------------------------------
  
    Calcurates the size of shared memory for the node table.
    
    This function calcurates the maximum memory size required for the node talbe.
    
    This is called from the following codes:
    
    \FuncRefHdr
		\FuncRef{CreateSharedMemoryAndSemaphores}{ipci.c}{
		  Used in estimating the size of the shared memory required in total.
		}\\ \hline
    \FuncRefTrailor
  
  \FUNC{PgxcNodeListAndCount()}	%--------------------------------------
  
    Updates node definitions in the shared memory tables with the system catalog data.
    
    This is called from the following codes:
    
    \FuncRefHdr
		\FuncRef{InitMultinodeExecutor}{pgxcnode.c}{
		  Called prior to calling \file{PgxcNodeGetOids()}
		}\\
		\FuncRef{PoolManagerCheckConnectionInfo}{poolmgr.c}{
		  Called before request to check the pooled connection.
		}\\
		\FuncRef{PoolManagerReloadConnectionInfo}{poolmgr.c}{
		  Called before request to reload the connection info.
		}\\
		\hline
    \FuncRefTrailor
  
  \FUNC{PgxcNodeGetOids()}	%--------------------------------------
  
    Builds a list of OIDs of coordinators and data nodes.
    
    This function creates a list of Oids of Coordinators and Datanodes currently exist in the shared memory,
	as well as number of Coordinators and Datanodes.
    
    This is called from the following codes (excluding call from pooler):
    
    \FuncRefHdr
		\FuncRef{pgxc_tablespace_size}{dbsize.c}{
		  Used in calcurating tablespace size
		}\\
		\FuncRef{pgxc_database_size}{dbsize.c}{
		  Used in calcurating database size
		}\\
		\FuncRef{pgxc_advisory_lock}{lockfuncs.c}{
		  Used in inquiring advisory locks
		}\\
		\hline
    \FuncRefTrailor
  
  \FUNC{PgxcNodeGetDefinition()}	%--------------------------------------
  
    Find node definition in the shared memory node table by oid.
    
    This function is called from the pooler.
  
  \FUNC{PgxcNodeAlter()}	%--------------------------------------
  
    Alter a PGXC node.
    
    This function is utility function which directly processes \texttt{ALTER NODE} statement as follows:
    
	\begin{enumerate}
		\item Opens the heap using \file{open_heap()},
	    \item Obtains a copy of current tuple of the target node from the system cache using \file{SearchSysCacheCopy1()},
		\item Updates it using \file{heap_modify_tuple()} and \file{simple_heap_update()},
		\item Updates the index of the system catalog using \file{CatalogUpdateIndexes()},
		\item Closes the heap using \file{close_heap()}.
	\end{enumerate}
    
    This is called from the following codes:
    
    \FuncRefHdr
		\FuncRef{standard_ProcessUtility}{utility.c}{
		  Used to process the {\tt AlterNodeStmt} type statement
		}\\ \hline
    \FuncRefTrailor
  
  \FUNC{PgxcNodeCreate()}	%--------------------------------------
  
    Adds a PGXC node.
    
    This function is a utility function which directly handles \texttt{CREATE NODE} statement as follows:
    
	\begin{enumerate}
		\item Generates new node id,
		\item Opens the heap using \file{open_heap()},
		\item Creates new tuple for the target node using \file{heap_from_tuple()},
		\item Insert it using \file{simple_heap_insert()},
		\item Updates index of the catalog using \file{CatalogUpdateIndexes()} and
		\item Closes the heap using \file{close_heap()}.
	\end{enumerate}
    
    This is called from the following codes:
    
    \FuncRefHdr
		\FuncRef{standard_ProcessUtility}{utility.c}{
		  Used to process the \file{CreateNodeStmt} type statement
		}\\ \hline
    \FuncRefTrailor
  
  \FUNC{PgxcNodeRemove()}	%--------------------------------------
  
    Remove a PGXC node
    
    This function is a utility function which directly processes \texttt{DROP NODE} statement as follows:
    
	\begin{enumerate}
		\item Opens the heap using \file{open_heap()},
		\item Obtains current tuple for the target node from the system cache using \file{SearchSysCache1()},
		\item Delete it using \file{simple_heap_delete()},
		\item Release the cached tuple using \file{ReleaseSysCache()()}
		\item and Closes heap using \file{close_heap()}.
	\end{enumerate}
    
    This is called from the following codes:
    
    \FuncRefHdr
		\FuncRef{standard_ProcessUtility}{utility.c}{
		  Used to process the \file{DropNodeStmt} type statement
		}\\ \hline
    \FuncRefTrailor



%========= SECTION SECTION ===================================================

\section{\label{sec:nodeGroup}Node Group}


%------- Subsec Subsec -----------------------------------------------------

\subsection{Node group management statement}

  To manage the cluster node group, there are \texttt{CREATE NODE GROUP} and
  \texttt{DROP NODE GROUP} statements.
  
  \XC{} extends the grammar to implement these statement. The grammar is defined in
  \file{src/backend/parser/gram.y}. And some changes made to support these statements.
  The changes are listed in Table \ref{tab:chggroupmgmtstmt}.
  
  \begin{table}[htp]
	  \begin{center}
		  \caption{\label{tab:chggroupmgmtstmt}Changes to support node group management statements}
		  \begin{tabular}{llp{0.5\hsize}} \hline
			  File & Function name & Description \\ \hline
			  \file{utility.c} & \file{standard\_ProcessUtility()} & {\raggedright
				  Changed to recognize the \file{CreateGroupStmt} and \file{DropGroupStmt} type statement
				  as a utility function and call appropriate function.} \\
			  \file{copyfunc.c} & \file{CopyObject()} & {\raggedright
				  Added support for \file{CreateGroupStmt} and \file{DropGroupStmt}.} \\
			  \file{equalfunc.c} & \file{equal()} & {\raggedright
				  Added support for \file{CreateGroupStmt} and \file{DropGroupStmt}.} \\
			  \file{utility.c} & \file{CreateCommandTag()} & {\raggedright
				  Added support for \file{CreateGroupStmt} and \file{DropGroupStmt}.} \\
			  \hline
			  \end{tabular}
	  \end{center}
  \end{table}
  
  The implementation of the statement function is described later.



%------- Subsec Subsec -----------------------------------------------------

\subsection{Group information catalog}

  These statements manipulates the system catalog \file{pgxc_group} internally.
  The definition of the table follows.
  You can find the column in the catalog corresponding to the options of the group
  management statement.
  
  % Program Listing -----------------------------------------------------------
  \begin{lstlisting}
                        Table "pg_catalog.pgxc_group"
    Column     |   Type    | Modifiers | Storage | Stats target | Description 
---------------+-----------+-----------+---------+--------------+-------------
 group_name    | name      | not null  | plain   |              | 
 group_members | oidvector | not null  | plain   |              | 
Indexes:
    "pgxc_group_name_index" UNIQUE, btree (group_name), tablespace "pg_global"
    "pgxc_group_oid" UNIQUE, btree (oid), tablespace "pg_global"
Has OIDs: yes
Tablespace: "pg_global"
  \end{lstlisting}
  % End program  Listing -------------------------------------------------------
  
  This catalog is defined in \file{src/include/catalog/pgxc_group.h},
  which are created by \file{initdb} through the BKI file.
  To get the node's attribute easily, utility functions listed in Table~
  \ref{tab:groupattrfunc} are implemented in \file{rc/backend/utils/cache/lsyscache.c}.
  For the more high level use, please see the following \textbf{Group Manager} subsection.
  
  \begin{table}[htp]
	  \begin{center}
		  \caption{\label{tab:groupattrfunc}Node group attribute query functions}
		  \begin{tabular}{lp{0.5\hsize}} \hline
			  Function name & Description \\ \hline
			  \file{get_pgxc_groupoid()} & Get the group OID by group name. \\
			  \file{get_pgxc_groupmembers()} & Get node OIDs of the group members by group OID. \\
			  \hline
		  \end{tabular}
	  \end{center}
  \end{table}


% - - - - subsubsection - - - - - - -  - - - - - - - - - - - - - - - - - - - - -

\subsection{Group manager}

  The group manager is implemented in \file{groupmgr.c}. This subsection describes
  APIs of the group manager.
  
  The group manager has following function:

  \begin{itemize}
  \item Manipulate node information in the \file{pgxc_group} catalog.
  \end{itemize}
  
  
  \FUNC{PgxcGroupCreate()}	%--------------------------------------
  
    Creates a PGXC node group.
    
    This function is a utility function which handles \texttt{CREATE NODE GROUP} statement directly as follows:
    
	\begin{enumerate}
		\item Builds a OID list of new group member node and converts the list to oid vector with \file{buildoidvector()},
		\item Opens the heap with \file{open_heap()},
		\item Creates new tuple for the target group with \file{heap_from_tuple()},
		\item Insert it with \file{simple_heap_insert()},
		\item Updates index of the catalog with \file{CatalogUpdateIndexes()},
	    \item and Closes heap with \file{close_heap()}.
	\end{enumerate}
    
    This is called from the following codes:
    
    \FuncRefHdr
		\FuncRef{standard_ProcessUtility}{utility.c}{
		  Used to process the \file{CreateGroupStmt} type statement
		}\\ \hline
    \FuncRefTrailor
  
  \FUNC{PgxcGroupRemove()}	%--------------------------------------
  
    Removes a PGXC node group.
    
    This function is a utility function which handles \texttt{DROP NODEnGROUP} statement directly as follows:
    
	\begin{enumerate}
		\item Opens the heap with \file{open_heap()},
		\item Obtains current tuple for the target node from the system cache with \file{SearchSysCache1()},
		\item Delete it with \file{simple_heap_delete()},
		\item Release the cached tuple with \file{ReleaseSysCache()}
		\item Closes heap with \file{close_heap()}.
	\end{enumerate}
    
    This is called from the following codes:
    
    \FuncRefHdr
  	  \FuncRef{standard_ProcessUtility}{utility.c}{
  		Used to process the \file{DropGrouptmt} type statement
  	  }\\
  	  \hline
    \FuncRefTrailor


%========= SECTION SECTION ===================================================

\section{\label{sec:extraTableAttr}Table Distribution Attributes}


%------- Subsec Subsec -----------------------------------------------------

\subsection{Table distribution statement}

  \XC{} stores extended table information around distribution, it is given in the table definition
  like ``\texttt{CREATE TABLE} .. \texttt{DISTRIBUTED BY} .. \texttt{NODE} ..''.
  Extended statements are listed below.
  
  \begin{itemize}
	  \item \texttt{CREATE TABLE}
	  \item \texttt{CREATE TABLE AS}
	  \item \texttt{ALTER TABLE}
  \end{itemize}
  
  To implement these extension, \XC{} extends the grammar.
  The grammar is defined in \file{src/backend/parser/gram.y}.
  
  \begin{table}[htp]
	  \begin{center}
		  \caption{\label{tab:}Changes of statements manipulates distribution information}
		  \scriptsize
		  \begin{tabular}{lllp{0.45\hsize}} \hline
			  Statement & File name & Function name & Description \\ \hline
			  \texttt{CREATE TABLE} & \file{tablecmds.c} & \file{DefineRelation()} &
			  		Changed to create \file{pgxc_class} entry using parsed tree node. \\
			  \texttt{CREATE TABLE AS} & - & - & Same as \texttt{CREATE TABLE} \\
			  \texttt{ALTER TABLE} & \file{tablecmd.c} & \file{AlterTableGetLockLevel()} &
			  		Set lock level for \file{AT_DistributeBy}, \file{AT_SubCluster}, \file{AT_AddNodeList}
					and \file{AT_DeleteNodeList} to exclusive \\
			  & \texttt{tablecmd.c} & \file{ATExecCmd()} &
			  		Adds support for \file{AT_DistributeBy}, \file{AT_SubCluster}, \file{AT_AddNodeList}
					and \file{AT_DeleteNodeList} \\
			  & & \file{ATPrepCmd()} &
			  		Adds support for \file{AT_DistributeBy}, \file{AT_SubCluster}, \file{AT_AddNodeList}
					and \file{AT_DeleteNodeList} \\
			  & \file{tablecmd.c} & \file{ATController()} &
			  		Changed to build redistribution commands and execute it. \\
			  \texttt{DROP TABLE} & \file{dependency.c}& \file{doDeletion()} &
			  		Added support for deletion of \file{OCLASS_PGXC_CALSS} object \\
			  \hline
		  \end{tabular}
	  \end{center}
  \end{table}
  
  
  
%------- Subsec Subsec -----------------------------------------------------

\subsection{Table distribution information catalog}
  
  These table distribution/redistribution statements manipulates the system catalog \file{pgxc_class} internally.
  The definition of the table follows. You can find the column in the catalog corresponding
  to the options of the extended statement.
  
  % Program listing -----------------------------------------------------------
  \begin{lstlisting}
                         Table "pg_catalog.pgxc_class"
     Column      |   Type    | Modifiers | Storage | Stats target | Description 
-----------------+-----------+-----------+---------+--------------+-------------
 pcrelid         | oid       | not null  | plain   |              | 
 pclocatortype   | "char"    | not null  | plain   |              | 
 pcattnum        | smallint  | not null  | plain   |              | 
 pchashalgorithm | smallint  | not null  | plain   |              | 
 pchashbuckets   | smallint  | not null  | plain   |              | 
 nodeoids        | oidvector | not null  | plain   |              | 
Indexes:
    "pgxc_class_pcrelid_index" UNIQUE, btree (pcrelid)
Has OIDs: no
  \end{lstlisting}
  % End program listing -------------------------------------------------------
  
  This catalog is defined in \file{src/include/catalog/pgxc_class.h}, 
  which are created by \file{initdb} through the BKI file.
  To get the node's attribute easily, utility functions listed in Table \ref{tab:xcclsattrfunc}
  are implemented in \file{src/backend/utils/cache/lsyscache.c}.
  Other high level functions are described in next section.
  
  \begin{table}[htp]
	  \begin{center}
		  \caption{\label{tab:xcclsattrfunc}Extended class attribute query functions}
		  \begin{tabular}{lp{0.5\hsize}} \hline
			  Function name & Description \\ \hline
			  \file{get_pgxc_classnodes()} & Gets the target node OIDs by table OID. \\
			  \hline
		  \end{tabular}
	  \end{center}
  \end{table}



%------- Subsec Subsec -----------------------------------------------------

\subsection{High-level functions for distributed table}


% - - - - subsubsection - - - - - - -  - - - - - - - - - - - - - - - - - - - - -

\subsubsection{\texttt{pgxc\_class.c}}

  This module supplies functions to manipulate \file{pgxc_class}.
  
  \FUNC{PgxcClassCreate()}	%--------------------------------------
  
    Creates a \file{pgxc_class} entry.
    
    This function manipulates \file{pgxc_class} system catalog information.
    
    This is called from the following codes:
    
    \FuncRefHdr
		\FuncRef{AddRelationDistribution}{heap.c}{
		  Used in adding to \file{pgxc_class} table
		}\\ \hline
    \FuncRefTrailor
  
  \FUNC{PgxcClassAlter()}	%--------------------------------------
  
    Modifies a \file{pgxc_class} entry with given data.
    
    This function manipulates \file{pgxc_class} system catalog information.
    
    This is called from the following codes:
    
    \FuncRefHdr
		\FuncRef{AtExecDistributeBy}{tablecmds.c}{
		  \vspace{3pt}
		}\\
		\FuncRef{AtExecSubCluster}{tablecmds.c}{
		  \vspace{3pt}
		}\\
		\FuncRef{AtExecAddNode}{tablecmds.c}{
		  \vspace{3pt}
		}\\
		\FuncRef{AtExecDeleteNode}{tablecmds.c}{
		  \vspace{3pt}
		}\\
		\hline
    \FuncRefTrailor
  
  \FUNC{RemovePgxcClass()}	%--------------------------------------
  
    Removes extended PGXC information.
    
    This function manipulates \file{pgxc_class} system  catalog information.
    
    This is called from the following codes:
    
    \FuncRefHdr
		\FuncRef{doDeletion}{dependency.c}{
		  Used in removing \file{pgxc_class} table
		}\\ \hline
    \FuncRefTrailor
  


