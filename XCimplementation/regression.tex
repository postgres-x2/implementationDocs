%%%%%%%%%%%%%%%%%%%%%%%%%%%%%%%%%%%%%%%%%%%%%%%%%%%%%%%%%%%%%%%%%%%%%%%%%%%%%%%%
%
% Chapter "Regression".
%
% Description of additional regression test for Postgres-XC.
%
%%%%%%%%%%%%%%%%%%%%%%%%%%%%%%%%%%%%%%%%%%%%%%%%%%%%%%%%%%%%%%%%%%%%%%%%%%%%%%%%


  This chapter describes changes to regression test.


%========= SECTION SECTION ===================================================

\section{General Changes}

  Most of the statements in the regression tests are used as is, or with minimum modification.
  General modification for \XC{} is as follows:
  
  \begin{itemize}
	  \item No \texttt{DISTRIBUTE BY} clause was given to \texttt{CREATE TABLE} or \texttt{ALTER TABLE}
	  		statement, if the first column is allowd as the distribution column.
	  \item If the first column is not allowd as the distribution column and another column
	  		is found suitable, \texttt{DISTRIBUTE BY HASH(}\textit{colname}\texttt{)}
			clause is added.
	  \item If no column is suitable for the distribution column, such table was defined as
	  		replicated table using \texttt{DISTRIBUTE BY REPLICATION}.
	  \item Because order of rows of \texttt{SELECT} statement or \texttt{RETURNING} clause depends upon
	  		the order of execution among datanodes, \texttt{ORDER BY} clause was added to make this
			order reproductive.
	  \item To make \texttt{EXPLAIN} result portable, \texttt{NODES OFF} and
			\texttt{NUM\_NODES OFF} options were added to make test result independent from
			the number of nodes and their names.
  \end{itemize}
  
  For specific test purpose, some tables in existing \PG{} regression test are created as
  replicated or round-robin distribution.
  
  Example of additional use of \texttt{ORDER BY} clause in \file{join.sql} is shown below.

  % Source listing -----------------
  \lstset{tabsize=4, xleftmargin=20pt, basicstyle=\ttfamily\small, breaklines=true}
  \begin{lstlisting}[frame=single]
 184 -- Outer joins
 185 -- Note that OUTER is a noise word
 186 --
 187 
 188 SELECT '' AS "xxx", *
 189   FROM J1_TBL LEFT OUTER JOIN J2_TBL USING (i)
 190   ORDER BY i, k, t;
 191 
 192 SELECT '' AS "xxx", *
 193   FROM J1_TBL LEFT JOIN J2_TBL USING (i)
 194   ORDER BY i, k, t;
 195 
 196 SELECT '' AS "xxx", *
 197   FROM J1_TBL RIGHT OUTER JOIN J2_TBL USING (i)
 198   ORDER BY i, j, k, t;
  \end{lstlisting}

  Example of additional option use in \texttt{EXPLAIN} in \file{join.sql} is shown below.

  \lstset{tabsize=4, xleftmargin=20pt, basicstyle=\ttfamily\small, breaklines=true}
  \begin{lstlisting}[frame=single]
 705 --
 706 -- test case where a PlaceHolderVar is propagated into a subquery
 707 --
 708 
 709 explain (num_nodes off, nodes off, costs off)
 710 select * from
 711   int8_tbl t1 left join
 712   (select q1 as x, 42 as y from int8_tbl t2) ss
 713   on t1.q2 = ss.x
 714 where
 715   1 = (select 1 from int8_tbl t3 where ss.y is not null limit 1)
 716 order by 1,2;
 717 
 718 select * from
 719   int8_tbl t1 left join
 720   (select q1 as x, 42 as y from int8_tbl t2) ss
 721   on t1.q2 = ss.x
 722 where
 723   1 = (select 1 from int8_tbl t3 where ss.y is not null limit 1)
 724 order by 1,2;
  \end{lstlisting}
    
  Please note that \texttt{ORDER BY} clause is given only to the subsequent \texttt{SELECT} statement
  because regression test needs to test the plan without \texttt{ORDER BY} close while subsequent
  \texttt{SELECT} statement needs to produce portable result.

  Please also not that \texttt{COSTS OFF} option in \texttt{EXPLAIN} statements are from vanilla
  \PG{} regression test to make explain result portable too.


%========= SECTION SECTION ===================================================

\section{Additional Test for \XC}

  Regression test prefixed with \file{xc_} is \XC-specific regression test.
  
  Table~ \ref{tab:xcRegression} describes each test.
  
  \begin{table}[htp]
	  \begin{center}
	  \caption{\label{tab:xcRegression}\XC-specific Regression Test}
		  \begin{tabular}{lp{0.7\hsize}} \hline
			  Test Name & Description \\ \hline
			  \file{xc_alter_table} & Checks \XC-specific behavior of \texttt{ALTER TABLE} statement,
			  						   such as dropping distribution column is not allowed.\\
			  \file{xc_constraints} & Checks constraint shippability in \XC{} for tables with
			  						  different distribution strategies.\\
			  \file{xc_create_function} & Defines a couple of function used by other \XC-specific
			  							   regression tests.\\
			  \file{xc_distkey} & Tests all the supported data types are working as distribution key.
			  					  Also verifies that comparison with a constant for equality is
								  optimized.\\
			  \file{xc_for_update} & Test of \texttt{FOR UPDATE} support in \XC.\\
			  \file{xc_FQS} & Test if fast query shipping works correctly in various distribution
			  				  strategy and statements.\\
			  \file{xc_FQS_join} & Dedicated test for join fast query shipping.\\
			  \file{xc_groupby} & Test of \texttt{GROUP BY} pushdown and cross node operation for
			  					  the combination of distributed and replicated tables.\\
			  \file{xc_having} & Tests \texttt{HAVING} clause handling for various combination of
			  					 table distribution.\\
			  \file{xc_limit} & Tests \texttt{LIMIT} and \texttt{OFFSET} clause push down.\\
			  \file{xc_misc} & Various feature test including plpgsql functions.\\
			  \file{xc_node} & Tests \XC{} node management statements.\\
			  \file{xc_params} & Tests non-simple variables (record types without \%rowtype descriptor) in
			  					 SQL statements inside a plpgsql function.\\
			  \file{xc_prepared_xacts} & Tests prepared transactions are working as expected.
			  							  Tests it is not prepared if a transaction is read only.\\
			  \file{xc_remote} & Tests \XC{} remote queries are working.  It is done disabling fast
			  					 query shipping to see standard planner works correctly.\\
			  \file{xc_returning} & Specific \texttt{RETURNING} clause test.\\
			  \file{xc_sequence} & Specific sequence test, including checking callback mechanisms
			  					   on GTM.\\
			  \file{xc_sort} & Test merge sort optimization in \XC.  This works to fetch ordered data
			  				   from datanodes and merge them at a coordinator.\\
			  \file{xc_temp} & Test temporary object behavior.\\
			  \file{xc_triggers} & Trigger test for various table distribution and trigger functions.\\
			  \file{xc_trigship} & Dedicated test for shippable and non-shippable trigger functions.\\
			  \hline
		  \end{tabular}
	  \end{center}
  \end{table}
