%%%%%%%%%%%%%%%%%%%%%%%%%%%%%%%%%%%%%%%%%%%%%%%%%%%%%%%%%%%%%%%%%%%%%%%%%%%%%
%
% Databae Object and DDL
%
%%%%%%%%%%%%%%%%%%%%%%%%%%%%%%%%%%%%%%%%%%%%%%%%%%%%%%%%%%%%%%%%%%%%%%%%%%%%%

%========= SECTION SECTION ===================================================

\section{DDL Propagation to Other Nodes}
  
  As mentioned in the architecture section~\ref{sec:ddlPropagation} at page~\pageref{sec:ddlPropagation},
  \XC{} propagates DDL execution to other
  node except for node management statements:
  \texttt{CREATE NODE}, \texttt{ALTER NODE}, \texttt{DROP NODE}, 
  \texttt{CREATE NODE GROUP}, and \texttt{DROP NODE GROUP}.
  
  In \PG, functions handling DDL statements start with those in
  \file{backend/tcop/utility.c}.
  Parsed DDL statements are fist handled by functions \file{ProcessUtility()}.
  If no hook is defined, then it is passed to \file{standard_ProcessUtility()}.
  If target object supports event triggers, then they are passed to
  \file{ProcessUtilitySlow()}.
  
  DDL propagation to other nodes is implemented in each DDL execution code.
  
  The following list shows an example for handling \texttt{CREATE TABLESPACE} statement
  in \file{standard_ProcessUtility()}.

  % Source list of the example: DDL propagation to other node
  \lstset{tabsize=4, xleftmargin=20pt, basicstyle=\ttfamily\scriptsize, breaklines=true}
  \begin{lstlisting}[frame=single, tabsize=4, language=C]
        case T_CreateTableSpaceStmt:
#ifdef PGXC
            if (IS_PGXC_COORDINATOR && !IsConnFromCoord())
#endif
            /* no event triggers for global objects */
            PreventTransactionChain(isTopLevel, "CREATE TABLESPACE");
            CreateTableSpace((CreateTableSpaceStmt *) parsetree);
#ifdef PGXC
            if (IS_PGXC_COORDINATOR && !IsConnFromCoord())
                ExecUtilityWithMessage(queryString, sentToRemote, false);
#endif
            break;
  \end{lstlisting}
  % End program listing -------------------------------------------------------

  The first block of the directive ``\file{# ifdef PGXC}'' tests if the current node is coordinator and
  the DDL is not from another coordinator.
  If so, then this is the root and has to check transaction chain.
  
  The second block of the directive ``\file{# ifdef PGXC}'' again tests if it the current node is a coordinator and
  the DDL is not from another coordinator.
  If so, this node has to take care of DDL propagation.
  Otherwise, it just concentrate on local handling.
  
  DDL propagation is handled in the same manner for other DDSs as well.


%========= SECTION SECTION ===================================================

\section{Additional Error Handling}

  If all the DDL can be handled within a transaction block and can handle abort correctly in vanilla \PG,
  \XC{} can use implicit 2PC to handle errors in other nodes.
  Unfortunately, some DDL such as \texttt{CRETE TABLESPACE} cannot be issued inside transaction block.
  \XC{} has to propagate such DDL too and has to complete the statement in atomic way.
  In other words, \XC{} has to guarantee that the DDL cannot be successful only partly in some nodes.
  \XC{} needs separate feature to maintain cluster-wide
  data integrity in such case, without using 2PC.
  
  To make this cleanup work, \XC{} defines internal structure \file{dbcleanup_info} of the type
  \file{abort_callback_type} inside \file{execRemote.c}.
  
  \file{set_dbcleanup_callback()} function registers cleanup function and cleanup data specific to each
  DDL which run only outside a transaction block.
  If this is registered, then \file{AtEOXact_DBCleanup()} will invoke it at the end of a transaction.


%========= SECTION SECTION ===================================================

\section{Additional Functions to handle DDL}

  \file{utility.c} is an entry point of most of the DDL handlers.
  To propagate DDL to other nodes, \XC{} implements several utility functions in this module.


%------- Func ----------------------------------------------------------

\FUNC{IsStmtAllowedInLockedMode()}

  Determines if a given statement can run within a transaction block.
  It is used to determine if dedicated error handling is needed.


%------- Func -----------------------------------------------------

\FUNC{ExecUtilityWithMessage()}

  This function performs a statement in a remote node within a transaction block.
  This handles error by attaching failed node name and by rethrowing it.


%------- Func -----------------------------------------------------

\FUNC{ExecUtilityStmtOnNodes()}

  This function executes a utility statement on nodes, including coordinators.


%------- Func -----------------------------------------------------

\FUNC{ExecUtilityFindNodes()}

  Determines the list of nodes to launch query on.


%------- Func -----------------------------------------------------

\FUNC{ExecUtilityFindNodesRelkind()}

  Determines which node a statement should be executed on the given relation.


%------- Func -----------------------------------------------------

\FUNC{GetNodesForCommentUtility()}

  Returns Object ID of object commented.


%------- Func -----------------------------------------------------

\FUNC{GetNodesForRulesUtility()}

  Gets the nodes to execute the given RULE related to a utility statement.


%------- Func -----------------------------------------------------

\FUNC{DropStmtPreTreatment()}

  Performs a pre-treatment of \texttt{DROP} statement on a remote coordinator.


%========= SECTION SECTION ===================================================

\section{\label{sec:tablespace}Tablespace}

%%%%%%%%%%%%%%%%%%%%%%%%%%%%%%%%%%%%%%%%%%%%%%%%%%%%%%%%%%%%%%%%%%%%%%%%%%%%%%%%%%%%%%%%%
%
% Tablespace Description
%
%%%%%%%%%%%%%%%%%%%%%%%%%%%%%%%%%%%%%%%%%%%%%%%%%%%%%%%%%%%%%%%%%%%%%%%%%%%%%%%%%%%%%%%%

This section describes \XC's tablespace implementation.

%------- Subsec Subsec -----------------------------------------------------

\subsection{Creating Tablespace}

  To create a tablespace, we need to specify an absolute directory path.
  In \XC, the tablespace need to be propagated to all the nodes, hence the directory path too.
  To maintain this syntax, \XC{} uses all this absolute path in all the nodes.
  
  This sounds reasonable if each node is configured in different server.
  There's no resource conflict.
  However, standard \XC{} configuration advises to install both coordinator and datanode
  at the same server and we need to resolve this conflict.
  
  Simple rule introduced here is to qualify tablespace directory path with the node name
  which does not conflict.
  It is safe enough because node name has to be unique throughout \XC{} cluster.
  
  The following shows additional code to do this in the function \file{CreateTableSpace()} in
  \file{tablespace.c}.

% Tablespace directory is qualified with the node name.
\lstset{tabsize=8, xleftmargin=20pt, basicstyle=\ttfamily\scriptsize, breaklines=true}
\begin{lstlisting}[frame=single, tabsize=4, language=C]
    /*
     * Check that location isn't too long. Remember that we're going to append
     * 'PG_XXX/<dboid>/<relid>.<nnn>'.  FYI, we never actually reference the
     * whole path, but mkdir() uses the first two parts.
     */
    if (strlen(location) + 1 + strlen(TABLESPACE_VERSION_DIRECTORY) + 1 +
#ifdef PGXC
        /*
         * In Postgres-XC, node name is added in the tablespace folder name to
         * insure unique names for nodes sharing the same server.
         * So real format is PG_XXX_<nodename>/<dboid>/<relid>.<nnn>''
         */
        strlen(PGXCNodeName) + 1 +
#endif
        OIDCHARS + 1 + OIDCHARS + 1 + OIDCHARS > MAXPGPATH)
        ereport(ERROR,
                (errcode(ERRCODE_INVALID_OBJECT_DEFINITION),
                 errmsg("tablespace location \"%s\" is too long",
                        location)));

\end{lstlisting}

  Node name is added to each tablespace directory so that it does not conflict if coordinator and
  datanode are configured in the same server and even if more than one coordinator or datanode is
  configured in the same saver.
  
  Before WAL records are written for this operation, \file{CreateTableSpace()} registers its
  cleanup function like:

 % Callback registration for additional DDL error handling to deal with
 % ones at remote nodes.
\begin{lstlisting}[frame=single, tabsize=4, language=C]
#ifdef PGXC
    /*
     * Even if we have succeeded, the transaction can be aborted because of
     * failure on other nodes. So register for cleanup.
     */
    set_dbcleanup_callback(createtbspc_abort_callback,
                          &tablespaceoid, sizeof(tablespaceoid));
#endif
\end{lstlisting}

  \file{createtbspc_abort_callback()} is implemented in \file{tablespace.c()} as well, where
  created subdirectory under the tablespace path is removed for cleanup.


%------- Subsec Subsec -----------------------------------------------------

\subsection{Modifying Tablespace}

  \file{AlterTableSpaceOptions()} is the handler of \texttt{ALTER TABLESPACE} statement.
  
  This can be performed within transaction block and there's no \XC-specific modification
  to this implementation.
  
  \subsection{Dropping Tablespace}
  
  \texttt{DROP TABLESPACE} cannot run within a transaction block.
  
  \file{DropTableSpace()} is the handler of this statement and there's no
  \XC-specific modification
  \footnote{
  	The reporter is not sure if this implementation is reasonable.
  	If \texttt{DROP TABLESPACE} fails in any of the nodes while propagating,
  	the operation has to be cleaned up, if vanilla \PG{} is doing so.
  	It may need more in-depth analysis of DROP TABLE failure handling in vanilla \PG.
  }.
  



%========= SECTION SECTION ===================================================

\section{\label{sec:mview}Materialized View}

%%%%%%%%%%%%%%%%%%%%%%%%%%%%%%%%%%%%%%%%%%%%%%%%%%%%%%%%%%%%%%%%%%%%%%%%%%%%%%%%%%%%
%
% Materialized View handling description
%
%%%%%%%%%%%%%%%%%%%%%%%%%%%%%%%%%%%%%%%%%%%%%%%%%%%%%%%%%%%%%%%%%%%%%%%%%%%%%%%%%%%%

  Just like usual views, materialized view is created at coordinator level, not datanode level,
  and is replicated among all the coordinators.
  When materialized view is created, originating coordinator collects all the rows and
  replicate them.
  
  When materialized view is refreshed, originating coordinator corrects all the rows,
  drops all the existing rows and then replicates new ones.
  

%------- Subsec Subsec -----------------------------------------------------

\subsection{Creating Materialized View}
  
  Materialized view is created by \texttt{CREATE MATERIALIZED VIEW} statement.
  Internally, this statement is handled as a variant of \texttt{CREATE TABLE AS} statement and
  handled by \file{ExecCreateTableAs()} in \file{createas.c}.
  
  The following is how this is handled in \file{utility.c}.

  % Materialized view DDL handling in XC ----------------------------------
  \lstset{tabsize=4, xleftmargin=20pt, basicstyle=\ttfamily\scriptsize, breaklines=true}
  \begin{lstlisting}[frame=single, tabsize=4, language=C]
            case T_CreateTableAsStmt:
                ExecCreateTableAs((CreateTableAsStmt *) parsetree,
                                  queryString, params, completionTag);
#ifdef PGXC
                /* Send CREATE MATERIALIZED VIEW command to all coordinators. */
                Assert(((CreateTableAsStmt *) parsetree)->relkind == OBJECT_MATVIEW);
                if (!((CreateTableAsStmt *) parsetree)->into->skipData && !IsConnFromCoord())
                    pgxc_send_matview_data(((CreateTableAsStmt *) parsetree)->into->rel,
                                            queryString);
                else
                    ExecUtilityStmtOnNodes(queryString, NULL, sentToRemote, false, EXEC_ON_COORDS, false);

#endif /* PGXC */
                break;
  \end{lstlisting}
  % End program listing ------------------------------------------------------

  Please note that \XC{} does not support \texttt{CREATE TABLE AS} statement and the above
  code is just for \texttt{CREATE MATERIALIZED VIEW} statement at present.
  
  Piece of the code at the parser (\file{gram.y}) is as follows:

  % Program Listing ----------------------------------------------------------
  \begin{lstlisting}[frame=single, tabsize=4, language=C]
/*****************************************************************************
 *
 *      QUERY :
 *              CREATE MATERIALIZED VIEW relname AS SelectStmt
 *
 *****************************************************************************/

CreateMatViewStmt:
        CREATE OptNoLog MATERIALIZED VIEW create_mv_target AS SelectStmt opt_with_data
                {
                    CreateTableAsStmt *ctas = makeNode(CreateTableAsStmt);
                    ctas->query = $7;
                    ctas->into = $5;
                    ctas->relkind = OBJECT_MATVIEW;
                    ctas->is_select_into = false;
                    /* cram additional flags into the IntoClause */
                    $5->rel->relpersistence = $2;
                    $5->skipData = !($8);
                    $$ = (Node *) ctas;
                }
        ;
  \end{lstlisting}
  % End Program Listing ------------------------------------------------------

  You will see that parse tree for \texttt{CREATE MATERIALIZED VIEW} statement is the same as
  \texttt{CREATE TABLE AS} statement.
  
  \texttt{CREATE TABLE AS} statement is blocked at present at \file{gram.y}.
  Therefore, \file{CreateTableAsStmt} node is used only for \texttt{CREATE MATERIALIZED VIEW}
  at present.


%------- Subsec Subsec -----------------------------------------------------

\subsection{Refreshing Materialized View}

  Contents of materialized views are refreshed by \texttt{REFRESH MATERIALIZED VIEW} statement.
  In \XC, materialized view refreshment causes all the old data are replaced with all the
  present data.
  
  This is handled by \PG~ backend function \file{ExecRefreshMatView()}.
  Its code is almost the same as vanilla \PG.
  Only one difference is if it is from another coordinator, that is, if new row data comes
  from originating coordinator, the data is handled using \texttt{COPY} protocol, not by
  running queries.
  
  Code snip in \file{ExecRefreshMatView()} is as follows:

% Code snip in Materialized View DDl handling.
\begin{lstlisting}[frame=single, tabsize=4, language=C]
#ifdef PGXC
    /*
	 * If the REFRESH command was received from other coordinator, it will also send
	 * the data to be filled in the materialized view, using COPY protocol.
	*/
    if (IsConnFromCoord())
	{
		Assert(IS_PGXC_COORDINATOR);
		pgxc_fill_matview_by_copy(dest, stmt->skipData, 0, NULL);
	}
    else
#endif /* PGXC */
\end{lstlisting}

  At the originating coordinator, \texttt{REFRESH MATERIALIZED VIEW} statement is handled locally
  first, and then the rows are propagated to other coordinators by using
  \file{pgxc_send_matview_data()} function.
  
  Implementation of two \XC-specific functions is as follows:

%- - - - - - - Func - - - - - - - - - - - - - - - - - - - - - - - - -

\FUNC{pgxc_send_matview_data()}

  This function is implemented in \file{matview.c}.
  It  opens specified materialized view, collect all the rows and send them to other
  coordinators using \texttt{COPY} command protocol.


%- - - - - - - Func - - - - - - - - - - - - - - - - - - - - - - - - -

\FUNC{pgxc_fill_matview_by_copy()}

  This function is implemented in \file{matview.c}.
  It receives table rows sent by \file{pgxc_send_matview_data()} and stores them in
  the target materialized view.


%- - - - - - - Subsubsection - - - - - - - - - - - - - - - - - - - - - - - - -

\subsection{Dropping Materialized View}

  It is handled by \file{ExecDropStmt()} function in \file{utility.c}.
  Additions in \XC{} is as follows:
  
  \begin{itemize}
	  \item Before removing local materialized view, \XC{} checks objects to be dropped.
	  		In materialized view, we don't have this yet.
	  \item After materialized view was removed locally, and if it is done in originating
	  		coordinator, then the DDL is propagated to other coordinators using
			\file{ExecUtilityStmtOnNodes()}, implemented in \file{utility.c}.
  \end{itemize}
  



%========= SECTION SECTION ===================================================

\section{\label{sec:updatableView}Automatic Updatable View}

%%%%%%%%%%%%%%%%%%%%%%%%%%%%%%%%%%%%%%%%%%%%%%%%%%%%%%%%%%%%%%%%%%%%%%%%%%%%%%%
%
% Automatic Updatable View description
%
%%%%%%%%%%%%%%%%%%%%%%%%%%%%%%%%%%%%%%%%%%%%%%%%%%%%%%%%%%%%%%%%%%%%%%%%%%%%%%

  An issue to support automatic updatable views is determining if a statement is
  updating distribution key, which is not allowed in \XC.
  
  Code changed a bit to handle this in view update.
  
  When a statement is rewritten which updates an updatable view, the result may
  include all the columns including distribution column, which is not updating anyway.
  
  The change determines this more strictly to allow such case.



%========= SECTION SECTION ===================================================

\section{\label{sec:trigger}Trigger}


%------- Subsec Subsec -----------------------------------------------------

\subsection{Trigger Syntax}

  Trigger syntax is defined in \file{gram.y}.
  There is no \XC-specific change in trngger syntax.


%------- Subsec Subsec -----------------------------------------------------

\subsection{Creating Trigger}

  \texttt{CREATE TRIGGER} statement is parsed into \file{CreateTrigStmt} structure and passed to
  \file{ProcessUtilitySlow()} function in \file{utility.c}.
  Here, after local handling has been done, the statement is propagated to other nodes where the base
  relation is defined.
  
  \file{commands/trigger.c} implements most of trigger DDL handler.
  \file{CreateTrigger()} is the main handler of {\tt C\tt REATE TRIGGER} statement.
  They have no \XC-specific changes.


%------- Subsec Subsec -----------------------------------------------------

\subsection{Changing Trigger definition}

  It is handled by \file{ExecRenameStmt()} in \file{alter.c}.
  Before this, the statement is propagated to other nodes where the base relation is defined
  \footnote{
	  We need to check if \texttt{ALTER TABLE} ... \texttt{ADD NODE} handle this correctly.
  }.
  
  In the case of \texttt{ALTER TRIGGER}, it is then passed to \file{remanetrig()} in \file{trigger.c}.
  
  They have no \XC-specific change.


%------- Subsec Subsec -----------------------------------------------------

\subsection{Dropping Trigger}

  This statement is handled by \file{ExecDropStmt()} in \file{utility.c} and then passed to
  \file{RemoveObjects()} in \file{dropcmds.c} before it is propagated to other nodes.
  
  \file{RemoveObjects()} does not have \XC-specific changes.


%------- Subsec Subsec -----------------------------------------------------

\subsection{Firing Trigger}

  Most of the changes needed to support triggers are in firing triggers, implemented in
  \file{trigger.c}.
  
  This section describes \XC-specific utility functions in this module and then describes
  changes to existing trigger firing functions.


% - - - - Func - - - - - - -  - - - - - - - - - - - - - - - - - - - - -

\FUNC{pgxc_should_exec_triggers()}

  Determines if all of the triggers for the relation should be executed
  here, on this node.
  On a coordinator, returns true if there is at least one non-shippable trigger for the relation
  that matches the given event, level and timing.
  (or for any local-only table for that matter), returns false if all of the matching triggers
  are shippable.
  
  PG behaviour is such that the triggers for the same table should be executed in
  alphabetical order.
  This make it essential to execute all the triggers on the same node, be it coordinator or
  datanode.
  So the idea used here is: if all matching triggers are shippable, they should be executed on local
  tables (i.e. on datanodes).
  Even if there is at least one single trigger that is not shippable, all the triggers should be fired on remote
  tables (i.e. on the coordinator).
  This ensures that either all the triggers are executed on coordinator, or all are executed on
  datanodes.


% - - - - Func - - - - - - -  - - - - - - - - - - - - - - - - - - - - -

\FUNC{pgxc_is_trigger_firable()}

This function is defined only to handle the special case if the trigger is an internal trigger.
Once we support global constraints, we should not handle this as a special case:
global constraint triggers would be executed just like normal triggers.
Internal triggers are internally created triggers for constraints such as foreign key or
unique constraints.
Currently we always execute an internal trigger on datanodes, assuming that the constraint
trigger function is always shippable to datanodes.
We can safely assume so because we disallow constraint creation for scenarios where the
constraint needs access to records on other nodes.


% - - - - Func - - - - - - -  - - - - - - - - - - - - - - - - - - - - -

\FUNC{pgxc_is_internal_trig_firable()}

  This function determines if a given internal trigger is firable at this node.


% - - - - Func - - - - - - -  - - - - - - - - - - - - - - - - - - - - -

\FUNC{pgxc_get_trigger_tuple()}

  Obtains tuple of the trigger target.


% - - - - Func - - - - - - -  - - - - - - - - - - - - - - - - - - - - -

\FUNC{pgxc_check_distcol_update()}

  Compares the distribution column values given to the function and error out if they
  are different.
  This is called to make sure triggers have not updated the distribution column.


% - - - - subsubsection - - - - - - -  - - - - - - - - - - - - - - - - - - - - -

\subsubsection{Other Trigger-firing Functions}

Other trigger-firint functions are modofied to determine if the trigger should be fired
in this node.
Fire it if yes.




%========= SECTION SECTION ===================================================

\section{\label{sec:eventTrigger}Event Trigger}

There are no \XC-specific changes in event trigger firing.

Changes a made to propagate all the DDLs to handle event trigger.

The changes are similar to other DDLs.




%========= SECTION SECTION ===================================================

\section{\label{sec:tempObject}Temporary Objects}

  The change has been made to allow temporary object usage in explicit 2PC transactions.
  
  The background that temporary object is not allowed in 2PC is that \textbf{PREPARED} transaction survives
  the session, while temporary objects do not.
  
  In contrary, implicit 2PCs do not survive the session.
  Even with crashes, \file{pgxc_clean} cleans up implicit 2PC transactions so that they do not
  survive.
  It is safe to allow temporary objects to be used in implicit 2PCs.
  
  Changes are done in \file{CommitTransaction()} in \file{xact.c}.
  
  The patch is as follows:

  % Patch to support temporary objects.
  \lstset{tabsize=4, xleftmargin=20pt, basicstyle=\ttfamily\scriptsize, breaklines=true}
  \begin{lstlisting}[frame=single, tabsize=4, language=C]
--- a/src/backend/access/transam/xact.c
+++ b/src/backend/access/transam/xact.c
@@ -2117,6 +2117,9 @@ CommitTransaction(void)
 {
	TransactionState s = CurrentTransactionState;
	TransactionId latestXid;
+#ifdef PGXC
+       bool            isImplicit = !(s->blockState == TBLOCK_PREPARE);
+#endif
							  
		ShowTransactionState("CommitTransaction");

@@ -2161,7 +2164,7 @@ CommitTransaction(void)
				*/
				if (IsOnCommitActions() || ExecIsTempObjectIncluded())
				{
-                       if (!EnforceTwoPhaseCommit)
+                       if (!EnforceTwoPhaseCommit || isImplicit)
							ExecSetTempObjectIncluded();
						else
							ereport(ERROR,
@@ -2655,7 +2658,11 @@ PrepareTransaction(void)
			* cases, such as a temp table created and dropped all within the
			* transaction.  That seems to require much more bookkeeping though.
			*/
+#ifdef PGXC
+       if (MyXactAccessedTempRel && !isImplicit)
+#else
		if (MyXactAccessedTempRel)
+#endif
			ereport(ERROR,
					(errcode(ERRCODE_FEATURE_NOT_SUPPORTED),
					 errmsg("cannot PREPARE a transaction that has operated on temporary tables")));
  \end{lstlisting}
