%
% Chapter "Documentation", description of how Postgres-XC reference document
% is composed.
%

	This chapter describes how \XC{} reference document is built.


\section{\XC{} Reference Document Source Structure}

	Because \XC{} inherits most of the spec from PostgreSQL, it is reasonable to prepare
	the reference manual using original PostgreSQL SGML files.
	With PostgreSQL document resource, we can build the following documents:

	\begin{enumerate}
		\item PDF (A4 and US Legal size)
		\item man pages
		\item html online document
		\item epub file (version 1.2 or later)
	\end{enumerate}

	Target should be as follows:

	\begin{enumerate}
		\item {\tt postgres-A4.pdf} or {\tt postgres-US.pdf}
		\item {\tt man}
		\item {\tt html}
		\item {\tt postgres.epub}
	\end{enumerate}

	In the case of {\tt html}, current \XC{} document handling framework allows
	to embed international characters like Japanese.

	We should be careful to make it easier to merge with later version of \PG{} SGML files.
	To achieve this, \XC{} reference document source allows to embed special "TAG" to distinguish
	what is common, what is \PG-specific and what is \XC-specific.
	Also, we may want to allow translations to different languages.
	To make it easier to handle as an external tool, \XC{} built dedicated (but somewhat general)
	tool to select what tags to be included.
	Tag is used to indicate what part of the original file should be taken or thrown away.
	Lines not enclosed with any such tags are common to all.
	So SGML file may look like...

	\begin{lstlisting}[frame=single]
...
<para>
<!## PG>
...
<!## end>
<!## XC>
...
<!## end>
</para>
	\end{lstlisting}

	You can nest this tag. With the nest, you can include different translations in a single file.

	This can be handled by a command {\tt makesgml}, which will be placed at
	\file{src/pgxc/tools/makesgml}.

	Makesgml can be invoked as follows:

	\begin{lstlisting}
makesgml -i inf -o outf -I include_tag ... -E exclude_tag ...
	\end{lstlisting}

	Each argument is optional and order of the argument is arbitrary.
	If you omit {\tt -i} option, it will read from stdin.
	If {\tt -o} is omitted, it will write to stdout.
	If input file include unspecified tags in the arguments, it will be treated as specified
	\file{-E}.

	All the sgml files from original PostgreSQL tarball will be renamed to sgmlin.
	Then it will be filtered by {\tt makesgml} and fed to original document build scripts.

\section{\XC{} Reference Documents}

	\XC{} reference document will be found in a separate PDF file.

