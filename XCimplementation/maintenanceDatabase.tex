%
% Chapter "Database Maintenance"
%
% Description of maintenance related commands.
%


\section{\label{sec:vacuum}Vacuum}

  Vacuum has two major classes: \texttt{VACUUM} and \texttt{Auto Vacuum} (\texttt{LAZY VACUUM}).
  There is a significant difference of Vacuum behavior from other statment.
  Datanodes obtain GXID and the snapshot for Vacuum directly from GTM by themselves.

  \texttt{VACUUM} is issed by DBA to collect garbages and analyze a database.
  Basically, \texttt{VACUUM} maintains the database just like \PG.
  But there's a bit difference to support \XC{} specific situation.
  
  \texttt{VACUUM} checks that the XID doesn't go backward after last vacuum to protect the database.
  But the XID could go out-of-sync in \XC{} when the datanode switch to global XID
  from local XID or it has been idle for a long time
  (and other nodes have been worked hard.)
  To fix this case, \XC{} allows rewinding XID in standalone mode.
  
  \texttt{Auto Vacuum} is invoked periodically specified by \file{autovacuum_naptime} GUC parameter.

  In \PG{}, \texttt{Auto Vacuum} is also performed each 65k transactions,
  but this feature does not work well in \XC{},
  because if the node does not involved in milestone transaction, he doesn't start \texttt{Auto Vacuum}.

  As \texttt{VACUUM} and \texttt{Auto Vacuum} shares the code, changes to \texttt{Auto Vacuum}
  are almost same as \texttt{VACUUM}.
  But please not that \texttt{Auto Vacuum} inquires special GXID that is not listed in global snapshot.
  \PG{} excludes \texttt{Auto Vacuum} transaction too.
  To do so, \texttt{Auto Vacuum} inquires a GXID to a GTM using dedicated message \file{TXN_BEGIN_GETGXID_AUTOVACUUM}.


\section{\label{sec:pgdump}Changes in \texttt{pg\_dump}}

  \XC{} has three extension/modification to \file{pg_dump} and \file{pg_dumpall} utility:
  
  \begin{itemize}
	  \item Including node definitions. (\file{pg_dumpall} only)
	  \item Including table distribution parameters into table definitions.
	  \item Obtain sequence values not only from sequence relations
            but GTM.
  \end{itemize}
  
  With \file{--dump-nodes} option, \file{pg_dump_all} includes nodes' definitions to
  dumped data by scanning \file{pg_catalog.pgxc_node} and \file{pg_catalog.pgxc_group}.
  
  Table distribution parameters mean \texttt{DISTRIBUTE BY} and \texttt{TO NODE} clauses.
  If a user want to include \texttt{TO NODE} clause into the dump file, \file{--include-nodes}
  option makes that clause.
  
  In \XC, locally cached values in sequence relations might not be latest values because
  the sequence values could have been modified by other coordinators.
  To backup the latest sequence values,
  \file{pg_dump} calls \file{pg_catalog.nextval}.

\section{\label{sec:redistrib}Table Redistribution}

  When we make a change in a distribution rule or the distribution target of a table,
  \XC{} redistribute existing data in the table.
  To redistribute data, the coordinator collects all the data from relevant datanodes,
  and then redistributes the data according to the new rule.
  
  The data redistribution is performed automatically by \texttt{ALTER TABLE} command.
  When \texttt{ALTER NODE} command detects the need of redistribution, it calls \file{BuildRedistribCommands()}
  to build redistribution state object.
  The redistribution state object includes SQL commands to redistribute,
  they are consist of \texttt{COPY TO}, \texttt{TRUNCATE} and \texttt{COPY FROM}.
  And \texttt{ALTER TABLE} calls \texttt{PGXCRedistribTable()} to  perform redistruibution.
  This function called twice: before and after update the catalog.
  Please visit \file{redistrib.c} for the logic of the redistribution.
