%
% Chapter "Postgres-XC source code tree structure"
%

%========= SECTION SECTION ===================================================

\section{\label{sec:additionalDir}Additional source directories}

  \XC{} source file tree structure follows \PG.
  Additional directories are shown in table \ref{tab:dirlist}.
  
  \begin{table}[htp]
	  \begin{center}
		  \caption{\label{tab:dirlist}Additional source directories for \XC}
		  \begin{tabular}{p{0.4\hsize}p{0.55\hsize}}\hline
				Directory&Description\\ \hline
				\file{contrib/pgxc_clean} & {\raggedright Additional module to cleanup errors.}\\
				\file{contrib/pgxc_ctl} & {\raggedright Additional module for \XC{} cluster configuration and operation.}\\
				\file{contrib/pgxc_ddl} & {\raggedright Additional module to propagate DDL execution to other nodes. }\\
				\file{contrib/pgxc_monitor} & {\raggedright Additional module to monitor if each node is running. }\\
				\file{doc-xc} & {\raggedright \XC~reference document.
								See chapter \ref{chap:documentation} at page \pageref{chap:documentation} for details.}\\
				\file{src/backend/pgxc/barrier} & {\raggedright Barrier module.   See chapter \ref{chap:architecture} for details.}\\
				\file{src/backend/pgxc/copy} & {\raggedright Copy command module.}\\
				\file{src/backend/pgxc/locator} & {\raggedright Module to locate target datanode of given row and table.}\\
				\file{src/backend/pgxc/nodemgr} & {\raggedright Module to read/write \file{pgxc_node} catalog.}\\
				\file{src/backend/pgxc/pool} & {\raggedright Connection pooling module between a coordinator and other nodes.}\\
				\file{src/backend/pgxc/xc_maintenance_mode} & {\raggedright Module to handle \file{xc_maintenance_mode} GUC parameter.}\\
				\file{src/backend/pgxc} & {\raggedright \XC~specific modules for coordinator/datanode which cannot be classified into existing directory.}\\
				\file{src/bin/gtm_ctl} & {\raggedright GTM and GTM proxy launcher.}\\
				\file{src/gtm} & {\raggedright Global transaction manager}\\
				\file{src/include/gtm} & {\raggedright Header files for gtm.}\\
				\file{src/include/pgxc} & {\raggedright Additional header files which cannot be classified into existing directory.}\\
				\file{src/pgxc/tools/makesgml} & {\raggedright SGML generator.  See \ref{chap:documentation} for details.}\\
				\file{src/pgxc} & {\raggedright Commonly used \XC-specific modules.}\\ \hline
		  \end{tabular}
	  \end{center}
  \end{table}


%========= SECTION SECTION ===================================================

\section{\label{sec:additionalFile}Additional source file for \XC}

  Additional \XC~ source files at existing \PG~source directory are shown in table \ref{tab:filelist}.
  Files sown in section\ref{chap:sourceTree}.\ref{sec:additionalDir} are not listed in this table.
  
  \begin{table}[htp]
	  \begin{center}
		  \caption{\label{tab:filelist}Additional source files for \XC}
		  \begin{tabular}{p{0.5\hsize}p{0.45\hsize}}\hline
				File&Description\\ \hline
				\file{src/backend/access/rmgrdesc/pgxcdesc.c} & Additional resource manager routines specific to \XC.\\
				\file{src/backend/access/transam/gtm.c} & Global transaction manager at coordinator/datanode side.\\
				\file{src/backend/catalog/pgxc_class.c} & \file{pgxc_class} system catalog handler.\\
				\file{src/backend/optimizer/path/pgxcpath.c} & \XC~additional module to find possible remote query paths.\\
				\file{src/backend/optimizer/plan/pgxcplan.c} & \XC~additional module of distributed query planner.\\
				\file{src/backend/optimizer/util/pgxcship.c} & \XC~additional module to evaluate expression shippability to remote nodes.\\
				\file{src/include/catalog/pgxc_class.h} & Definition of \file{pgxc_class} system catalog.\\
				\file{src/include/catalog/pgxc_group.h} & Definition of \file{pgxc_group} system catalog.\\
				\file{src/include/catalog/pgxc_node.h} & Definition of \file{pgxc_node} system catalog.\\
				\file{src/include/optimizer/pgxcplan.h} & Additional definition for the planner.\\
				\file{src/include/optimizer/pgxcship.h} & Additional definition for evaluation of expression shippability to datanodes.\\
				\hline
		  \end{tabular}
	  \end{center}
  \end{table}


%========= SECTION SECTION ===================================================

\section{\label{sec:existingFiles}Modification to existing files}

  Modification of existing \PG~source files are made as follows:
  
  \begin{itemize}
	  \item For *{\tt.c} and *{\tt.h} files, \XC-specific lines are indicated by C-pre-compiler
	  		directive using {\tt PGXC} label.  An example is given in Figure \ref{fig:PGXCDirective}.
	  \item For *{\tt.l} and *{\tt.y} files, flex and bison does not provide directive as used in C
	  		source file.  Modifications were done directly to these files.
	  \item For document files, all existing sgml files are renamed into sgmlin files to include
	  		directives to indicate \XC~specific description.
	  		See Chapter \ref{chap:documentation} at page \pageref{chap:documentation}for details.
  \end{itemize}
  
  \begin{figure}
	  % Source listing ----------------------------------------------
	  \begin{lstlisting}[frame=single]
#ifdef PGXC
    while ((c = getopt_long(argc, argv,
					        "ih:knvp:dSNc:j:Crs:t:T:U:lf:D:F:M:",
							long_options, &optindex)) != -1)
#else
    while ((c = getopt_long(argc, argv,
					        "ih:nvp:dqSNc:j:Crs:t:T:U:lf:D:F:M:",
							long_options, &optindex)) != -1)
#endif
	  \end{lstlisting}
	  % End source listing ------------------------------------------
	  \begin{center}
		  \caption{\label{fig:PGXCDirective}Example of C source code modification}
	  \end{center}
  \end{figure}
  
%========= SECTION SECTION ===================================================

\section{\label{sec:numOfLines}Number of lines of the source}

    Source code size of \XC{} was estimated by counting lines of additional souce code of
    \XC{} to corresponding \PG{} source by using \texttt{diff}.
  
    In counting the lines of the code, the following directories and files are excluded:

    \begin{itemize}
  		\item Copy from \PG{}, such as libpq at gtm.
		\item Reference documents.
		\item Regression test source/expected result.
	\end{itemize}

	\XC{} source code commit is \file{REL1_2_STABLE}, while \PG{} source code commit
	is \file{f5f21315d25ffcbfe7c6a3fa6ffaad54d31bcde0}.

	As the result, additional source code lines from correponding \PG{} is about
	ninety-thousand lines.

