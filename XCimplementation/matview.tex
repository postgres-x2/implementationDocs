%%%%%%%%%%%%%%%%%%%%%%%%%%%%%%%%%%%%%%%%%%%%%%%%%%%%%%%%%%%%%%%%%%%%%%%%%%%%%%%%%%%%
%
% Materialized View handling description
%
%%%%%%%%%%%%%%%%%%%%%%%%%%%%%%%%%%%%%%%%%%%%%%%%%%%%%%%%%%%%%%%%%%%%%%%%%%%%%%%%%%%%

  Just like usual views, materialized view is created at coordinator level, not datanode level,
  and is replicated among all the coordinators.
  When materialized view is created, originating coordinator collects all the rows and
  replicate them.
  
  When materialized view is refreshed, originating coordinator corrects all the rows,
  drops all the existing rows and then replicates new ones.
  

%------- Subsec Subsec -----------------------------------------------------

\subsection{Creating Materialized View}
  
  Materialized view is created by \texttt{CREATE MATERIALIZED VIEW} statement.
  Internally, this statement is handled as a variant of \texttt{CREATE TABLE AS} statement and
  handled by \file{ExecCreateTableAs()} in \file{createas.c}.
  
  The following is how this is handled in \file{utility.c}.

  % Materialized view DDL handling in XC ----------------------------------
  \lstset{tabsize=4, xleftmargin=20pt, basicstyle=\ttfamily\scriptsize, breaklines=true}
  \begin{lstlisting}[frame=single, tabsize=4, language=C]
            case T_CreateTableAsStmt:
                ExecCreateTableAs((CreateTableAsStmt *) parsetree,
                                  queryString, params, completionTag);
#ifdef PGXC
                /* Send CREATE MATERIALIZED VIEW command to all coordinators. */
                Assert(((CreateTableAsStmt *) parsetree)->relkind == OBJECT_MATVIEW);
                if (!((CreateTableAsStmt *) parsetree)->into->skipData && !IsConnFromCoord())
                    pgxc_send_matview_data(((CreateTableAsStmt *) parsetree)->into->rel,
                                            queryString);
                else
                    ExecUtilityStmtOnNodes(queryString, NULL, sentToRemote, false, EXEC_ON_COORDS, false);

#endif /* PGXC */
                break;
  \end{lstlisting}
  % End program listing ------------------------------------------------------

  Please note that \XC{} does not support \texttt{CREATE TABLE AS} statement and the above
  code is just for \texttt{CREATE MATERIALIZED VIEW} statement at present.
  
  Piece of the code at the parser (\file{gram.y}) is as follows:

  % Program Listing ----------------------------------------------------------
  \begin{lstlisting}[frame=single, tabsize=4, language=C]
/*****************************************************************************
 *
 *      QUERY :
 *              CREATE MATERIALIZED VIEW relname AS SelectStmt
 *
 *****************************************************************************/

CreateMatViewStmt:
        CREATE OptNoLog MATERIALIZED VIEW create_mv_target AS SelectStmt opt_with_data
                {
                    CreateTableAsStmt *ctas = makeNode(CreateTableAsStmt);
                    ctas->query = $7;
                    ctas->into = $5;
                    ctas->relkind = OBJECT_MATVIEW;
                    ctas->is_select_into = false;
                    /* cram additional flags into the IntoClause */
                    $5->rel->relpersistence = $2;
                    $5->skipData = !($8);
                    $$ = (Node *) ctas;
                }
        ;
  \end{lstlisting}
  % End Program Listing ------------------------------------------------------

  You will see that parse tree for \texttt{CREATE MATERIALIZED VIEW} statement is the same as
  \texttt{CREATE TABLE AS} statement.
  
  \texttt{CREATE TABLE AS} statement is blocked at present at \file{gram.y}.
  Therefore, \file{CreateTableAsStmt} node is used only for \texttt{CREATE MATERIALIZED VIEW}
  at present.


%------- Subsec Subsec -----------------------------------------------------

\subsection{Refreshing Materialized View}

  Contents of materialized views are refreshed by \texttt{REFRESH MATERIALIZED VIEW} statement.
  In \XC, materialized view refreshment causes all the old data are replaced with all the
  present data.
  
  This is handled by \PG~ backend function \file{ExecRefreshMatView()}.
  Its code is almost the same as vanilla \PG.
  Only one difference is if it is from another coordinator, that is, if new row data comes
  from originating coordinator, the data is handled using \texttt{COPY} protocol, not by
  running queries.
  
  Code snip in \file{ExecRefreshMatView()} is as follows:

% Code snip in Materialized View DDl handling.
\begin{lstlisting}[frame=single, tabsize=4, language=C]
#ifdef PGXC
    /*
	 * If the REFRESH command was received from other coordinator, it will also send
	 * the data to be filled in the materialized view, using COPY protocol.
	*/
    if (IsConnFromCoord())
	{
		Assert(IS_PGXC_COORDINATOR);
		pgxc_fill_matview_by_copy(dest, stmt->skipData, 0, NULL);
	}
    else
#endif /* PGXC */
\end{lstlisting}

  At the originating coordinator, \texttt{REFRESH MATERIALIZED VIEW} statement is handled locally
  first, and then the rows are propagated to other coordinators by using
  \file{pgxc_send_matview_data()} function.
  
  Implementation of two \XC-specific functions is as follows:

%- - - - - - - Func - - - - - - - - - - - - - - - - - - - - - - - - -

\FUNC{pgxc_send_matview_data()}

  This function is implemented in \file{matview.c}.
  It  opens specified materialized view, collect all the rows and send them to other
  coordinators using \texttt{COPY} command protocol.


%- - - - - - - Func - - - - - - - - - - - - - - - - - - - - - - - - -

\FUNC{pgxc_fill_matview_by_copy()}

  This function is implemented in \file{matview.c}.
  It receives table rows sent by \file{pgxc_send_matview_data()} and stores them in
  the target materialized view.


%- - - - - - - Subsubsection - - - - - - - - - - - - - - - - - - - - - - - - -

\subsection{Dropping Materialized View}

  It is handled by \file{ExecDropStmt()} function in \file{utility.c}.
  Additions in \XC{} is as follows:
  
  \begin{itemize}
	  \item Before removing local materialized view, \XC{} checks objects to be dropped.
	  		In materialized view, we don't have this yet.
	  \item After materialized view was removed locally, and if it is done in originating
	  		coordinator, then the DDL is propagated to other coordinators using
			\file{ExecUtilityStmtOnNodes()}, implemented in \file{utility.c}.
  \end{itemize}
  
