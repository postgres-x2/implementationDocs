%%%%%%%%%%%%%%%%%%%%%%%%%%%%%%%%%%%%%%%%%%%%%%%%%%%%%%%%%%%%%%%%%%%%%%%%%%%%%%%%%%%%%%%%%
%
% Tablespace Description
%
%%%%%%%%%%%%%%%%%%%%%%%%%%%%%%%%%%%%%%%%%%%%%%%%%%%%%%%%%%%%%%%%%%%%%%%%%%%%%%%%%%%%%%%%

This section describes \XC's tablespace implementation.

%------- Subsec Subsec -----------------------------------------------------

\subsection{Creating Tablespace}

  To create a tablespace, we need to specify an absolute directory path.
  In \XC, the tablespace need to be propagated to all the nodes, hence the directory path too.
  To maintain this syntax, \XC{} uses all this absolute path in all the nodes.
  
  This sounds reasonable if each node is configured in different server.
  There's no resource conflict.
  However, standard \XC{} configuration advises to install both coordinator and datanode
  at the same server and we need to resolve this conflict.
  
  Simple rule introduced here is to qualify tablespace directory path with the node name
  which does not conflict.
  It is safe enough because node name has to be unique throughout \XC{} cluster.
  
  The following shows additional code to do this in the function \file{CreateTableSpace()} in
  \file{tablespace.c}.

% Tablespace directory is qualified with the node name.
\lstset{tabsize=8, xleftmargin=20pt, basicstyle=\ttfamily\scriptsize, breaklines=true}
\begin{lstlisting}[frame=single, tabsize=4, language=C]
    /*
     * Check that location isn't too long. Remember that we're going to append
     * 'PG_XXX/<dboid>/<relid>.<nnn>'.  FYI, we never actually reference the
     * whole path, but mkdir() uses the first two parts.
     */
    if (strlen(location) + 1 + strlen(TABLESPACE_VERSION_DIRECTORY) + 1 +
#ifdef PGXC
        /*
         * In Postgres-XC, node name is added in the tablespace folder name to
         * insure unique names for nodes sharing the same server.
         * So real format is PG_XXX_<nodename>/<dboid>/<relid>.<nnn>''
         */
        strlen(PGXCNodeName) + 1 +
#endif
        OIDCHARS + 1 + OIDCHARS + 1 + OIDCHARS > MAXPGPATH)
        ereport(ERROR,
                (errcode(ERRCODE_INVALID_OBJECT_DEFINITION),
                 errmsg("tablespace location \"%s\" is too long",
                        location)));

\end{lstlisting}

  Node name is added to each tablespace directory so that it does not conflict if coordinator and
  datanode are configured in the same server and even if more than one coordinator or datanode is
  configured in the same saver.
  
  Before WAL records are written for this operation, \file{CreateTableSpace()} registers its
  cleanup function like:

 % Callback registration for additional DDL error handling to deal with
 % ones at remote nodes.
\begin{lstlisting}[frame=single, tabsize=4, language=C]
#ifdef PGXC
    /*
     * Even if we have succeeded, the transaction can be aborted because of
     * failure on other nodes. So register for cleanup.
     */
    set_dbcleanup_callback(createtbspc_abort_callback,
                          &tablespaceoid, sizeof(tablespaceoid));
#endif
\end{lstlisting}

  \file{createtbspc_abort_callback()} is implemented in \file{tablespace.c()} as well, where
  created subdirectory under the tablespace path is removed for cleanup.


%------- Subsec Subsec -----------------------------------------------------

\subsection{Modifying Tablespace}

  \file{AlterTableSpaceOptions()} is the handler of \texttt{ALTER TABLESPACE} statement.
  
  This can be performed within transaction block and there's no \XC-specific modification
  to this implementation.
  
  \subsection{Dropping Tablespace}
  
  \texttt{DROP TABLESPACE} cannot run within a transaction block.
  
  \file{DropTableSpace()} is the handler of this statement and there's no
  \XC-specific modification
  \footnote{
  	The reporter is not sure if this implementation is reasonable.
  	If \texttt{DROP TABLESPACE} fails in any of the nodes while propagating,
  	the operation has to be cleaned up, if vanilla \PG{} is doing so.
  	It may need more in-depth analysis of DROP TABLE failure handling in vanilla \PG.
  }.
  
