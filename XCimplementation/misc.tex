%%%%%%%%%%%%%%%%%%%%%%%%%%%%%%%%%%%%%%%%%%%%%%%%%%%%%%%%%%%%%%%%%%%%%%%%%%%%%%%%
%
% Miscellaneous Feature section.   Describes additional GUCS to Postgres-XC.
%
%%%%%%%%%%%%%%%%%%%%%%%%%%%%%%%%%%%%%%%%%%%%%%%%%%%%%%%%%%%%%%%%%%%%%%%%%%%%%%%%


%========= SECTION SECTION ===================================================

\section{\label{sec:GUC}Additional \texttt{postgresql.conf} configuration parameters}

  This section describes additional \file{postgresql.conf} configuration parameters specific to \XC.


%------- Subsec Subsec -----------------------------------------------------

\subsection*{\label{subsec:enableFQS}\texttt{enable\_fast\_query\_shipping}}

  This is boolean parameter to specify if fast query shipping is enabled.
  Usually this should be {\tt ON}.


%------- Subsec Subsec -----------------------------------------------------

\subsection*{\label{subsec:enableRemotegroup}\texttt{enable\_remotegroup}}

  This is boolean parameter to specify if group-by push down to remote nodes is enabled.
  Usually this should be {\tt ON}.


%------- Subsec Subsec -----------------------------------------------------

\subsection*{\label{subsec:enableRemotejoin}\texttt{enable\_remotejoin}}

  This is boolean parameter to specify if join push down to remote nodes is enabled.
  Usually this should be {\tt ON}.


%------- Subsec Subsec -----------------------------------------------------

\subsection*{\label{subsec:enableRemotelimit}\texttt{enable\_remotelimit}}

  This is boolean parameter to specify if limit pushdown to remote nodes is enabled.
  Usually this should be {\tt ON}.


%------- Subsec Subsec -----------------------------------------------------

\subsection*{\label{subsec:enableRemotesort}\texttt{enable\_remotesort}}

  This is boolean parameter to specify if ORDER BY pushdown to remote node is enabled.
  Usually this should be {\tt ON}.


%------- Subsec Subsec -----------------------------------------------------

\subsection*{\label{subsec:enforceTwoPhaseCommit}\texttt{enforce\_two\_phase\_commit}}

  This is boolean parameter to specify if two phase commit protocol is used for
  write transactions more than two nodes are involved.


%------- Subsec Subsec -----------------------------------------------------

\subsection*{\label{subsec:gtmBackupBarrier}\texttt{gtm\_backup\_barrier}}

  This is boolean parameter to specify if GTM restart point is backed up for barrier.


%------- Subsec Subsec -----------------------------------------------------

\subsection*{\label{subsec:gtmHost}\texttt{gtm\_host}}

  This is character string parameter to specify host name of GTM.
  If you configure GTM proxy, you should specify host name of GTM proxy.


%------- Subsec Subsec -----------------------------------------------------

\subsection*{\label{subsec:gtmPort}\texttt{gtm\_port}}

  This is integer parameter to specify port number of GTM.
  If you configure GTM proxy, you should specify port number of GTM proxy.


%------- Subsec Subsec -----------------------------------------------------

\subsection*{\label{subsec:maxCoordinators}\texttt{max\_coordinators}}

  This is integer parameter to specify the maximum number of coordinators in the cluster.


%------- Subsec Subsec -----------------------------------------------------

\subsection*{\label{subsec:maxDatanodes}\texttt{max\_datanodes}}

  This is numeric parameter to specify the maximum number of datanodes in the cluster.


%------- Subsec Subsec -----------------------------------------------------

\subsection*{\label{subsec:maxPoolSize}\texttt{max\_pool\_size}}

  This is numeric parameter to specify the maximum number of pooled connection in the pooler.


%------- Subsec Subsec -----------------------------------------------------

\subsection*{\label{subsec:minPoolSize}\texttt{min\_pool\_size}}

  This is numeric parameter to specify the minimum number of pooled connection in the pooler.


%------- Subsec Subsec -----------------------------------------------------

\subsection*{\label{subsec:persistentDatanodeConnections}\texttt{persistent\_datanode\_connections}}

  This is boolean parameter to specify if the connections from coordinator to datanodes should keep assigned.


%------- Subsec Subsec -----------------------------------------------------

\subsection*{\label{subsec:pgxcNodename}\texttt{pgxc\_node\_name}}

  This is character string parameter to specify the node name of itself.

\subsection*{\label{subsec:poolerPort}\texttt{pooler\_port}}

This is numeric parameter to specify the port number of the pooler.


%------- Subsec Subsec -----------------------------------------------------

\subsection*{\label{subsec:remoteType}\texttt{remotetype}}

  Used to identify what is connecting to the backend.
  Usually, do not modify this parameter.


%------- Subsec Subsec -----------------------------------------------------

\subsection*{\label{subsec:requireRepTablePkey}\texttt{require\_replicated\_table\_pkey}}

  Boolean parameter to specify if it is not allowed replicated tables without primary key or another unique key combination involved.
  If this is turned on and no such unique key is not involved, the statement fails.


%------- Subsec Subsec -----------------------------------------------------

\subsection*{\label{subsec:maintenanceMode}\texttt{xc\_maintenance\_mode}}

  Boolean parameter to control if write operation is allowed in {\tt EXECUTE DIRECT}.
  This parameter cannot turn on in {\tt postgresql.conf}.
  Only a superuser can turn it on in a session.


%------- Subsec Subsec -----------------------------------------------------

\section{\label{sec:syntax}Additional SQL syntax for \XC}

  The following lists \XC-specific SQL statement syntax.
  Refer to \XC~ documentation for details.
  
  \begin{itemize}
	  \item \texttt{ALTER NODE} statement.
	  \item \texttt{DISTRIBUTE BY} clause in \texttt{ALTER TABLE} statement.
	  \item\texttt{CLEAN CONNECTION} statement.
	  \item Collection function in \texttt{CREATE AGGREGATE} statement.
	  \item \texttt{CREATE BARRIER} statement.
	  \item \texttt{CREATE NODE} statement.
	  \item \texttt{CREATE NODE GROUP} statement.
	  \item Additional options for \texttt{EXPLAIN} statement.
	  \item \texttt{DISTRIBUTE BY} clause in \texttt{CREATE TABLE} statement.
	  \item \texttt{DROP NODE} statement.
	  \item \texttt{DROP NODE GROUP} statement.
  \end{itemize}
